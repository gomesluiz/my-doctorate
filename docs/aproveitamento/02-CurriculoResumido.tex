\documentclass[margin, 10pt]{res}
\usepackage[utf8]{inputenc}
\usepackage{helvet}
\usepackage{url}
\usepackage{fancyhdr}
 
\pagestyle{plain}
 


\begin{document}

\moveleft.5\hoffset\centerline{\bf Currículo Resumido} 
\moveleft.5\hoffset\centerline{\large\bf Luiz Alberto Ferreira Gomes}
\moveleft.5\hoffset\centerline{44 anos, casado}
\moveleft.5\hoffset\centerline{Rua Antônio Luiz Pinto, 60/5}
\moveleft.5\hoffset\centerline{37701.275 - Poços de Caldas, MG }
\moveleft.5\hoffset\centerline{(035) 8819-9005 ou (035)3715-6229}
\moveleft.5\hoffset\centerline{luizgomes@pucpcaldas.br} 
\moveleft.5\hoffset\centerline{Lattes: \url{http://lattes.cnpq.br/5732116193852474}}

\begin{resume}
  \section{FORMAÇÃO ACADÊMICA}
  {\sl Mestrado em Ciência da Computação}\hfill 2001-2003\\
  Universidade Estadual de Campinas, UNICAMP, Brasil. 
  \begin{itemize} \itemsep -2pt 
    \item Dissertação, defendida em abril de 2003, com o título "Comércio Eletrônico: Uma Análise da Aplicabilidade de Modelos de Qualidade",
    sob a orientação do prof. PhD Mario Lúcio Côrtes.
  \end{itemize}

  {\sl Bacharelado em Ciência da Computação}\hfill 1993-1996\\
  Universidade Federal de Ouro Preto, UFOP, Brasil. 
  \begin{itemize} \itemsep -2pt 
    \item Monografia, defendida em dezembro de 1996, com o título "Conexão de Banco de Dados com o Ambiente Web", sob a orientação do prof. Dr. 
    Carlos Frederico Marcelo da Cunha Cavalcanti.  
  \end{itemize}

  \section{EXPERIÊNCIA PROFISSIONAL}
  {\sl Professor Assistente} \hfill 2002-Atual \\
  Pontifícia Univerdade Católica de Minas Gerais, Departamento de Ciência da Computação, Poços de Caldas, MG. 
    \begin{itemize} \itemsep -2pt % Reduce space between items
      \item Responsável pelas disciplinas Análise e Projeto de Sistemas I e II, Engenharia de Software e Segurança e Auditoria de Sistemas.
      \item Orientação dos projetos de iniciação científica na áreas de arquitetura orientada a serviços, linhas de produtos de software e aspectos.     
   \end{itemize}
   
  {\sl Coordenador de Cursos de Pós-Graduação} \hfill 2009-2013 \\ 
  Pontifícia Univerdade Católica de Minas Gerais, Departamento de Ciência da Computação, Poços de Caldas, MG. 
    \begin{itemize} \itemsep -2pt % Reduce space between items
      \item Responsável pelos cursos de Engenharia de Redes e Sistemas Distribuídos e o de Tecnologias da Informação e Comunicação.
    \end{itemize}
  
  {\sl Analista de Sistemas } \hfill 2000-2001 \\
  UPTIME Sistemas, Belo Horizonte, MG.
    \begin{itemize} \itemsep -2pt % Reduce space between items
      \item Participação em projetos de sistema de software para "\textit{billing}" de serviços de dados para celulares pré-pagos de 
	clientes da Telemig Celular(adquirida pela Vivo).
    \end{itemize}

  {\sl Analista de Sistemas } \hfill 1998-2000 \\
  DEZ Tecnologia, Belo Horizonte, MG.
    \begin{itemize} \itemsep -2pt % Reduce space between items
      \item Participação em projeto de migração, em parceria com a Oracle do Brasil, de mais de 60 banco de dados da empresa Usiminas $\textendash$ incluindo procedimentos armazenados  $\textendash$ do 
      SGBD Sybase para o SGBD Oracle.
    \end{itemize}

  {\sl Analista de Sistemas } \hfill 1997-1998 \\
  Fóton Informática SA, Brasília, DF.
    \begin{itemize} \itemsep -2pt % Reduce space between items
      \item Participação em projetos de desenvolvimento de softwares autorizadores de transações bancárias para as seguintes instituições: Banco da Amazônia (adquirido pelo Itaú),
      Banco do Estado de Minas Gerais (adquirido pelo Itaú) e Banco Mercantil do Brasil.
    \end{itemize}

  \section{ARTIGOS PUBLICADOS}
  \begin{itemize}
    \item GOMES, LUIZ ; BRAGA, R. T. V. . Uso de Padrões em Linhas de Produtos de Software: Uma Revisão Sistemática. In: 7th Latin American Conference on 
      Pattern Languages of Programming, 2008, Fortaleza. Proceedings of 7th Latin American Conference on Pattern Languages of Programming. Fortaleza: UECE, 2008. 
    \item FRITZKE JR., U. ; GOMES, LUIZ ; SILVA, D. L. ; MORAES, D.M . A meta protocol for adaptive replication control on mobile databases. In: XV Encontro de 
      Iniciação Científica, 2008, Belo Horizonte. XV Seminário de Iniciação Científica: Destaques 2007. Belo Horizonte: Ed. PUC Minas, 2007. v. 2. p. 479-502. 
    \item FRITZKE JR., U. ; GOMES, LUIZ ; SILVA, D. L. ; MORAES, D.M . A Meta Protocol for Adaptable Mobile Replicated Databases. In: VIII Workshop de Teste e Tolerância a Falhas, 
      2007, Belém - Pará. SBRC 2007 - VII Workshop de Teste e Tolerância a Falhas (WTF 2007). Belém: UFPA, 2007. p. 173-186. 
    \item TAVARES, T. C. ; GOMES, LUIZ ; FRITZKE JR., U. . Middleware Adaptável para Banco de Dados Móveis Utilizando Aspectos. In: III Workshop Brasileiro de 
      Desenvolvimento de Software Orientado a Aspectos, 2006, Florianópolis. Anais III Workshop Brasileiro de Desenvolvimento de Software Orientado a Aspectos 
      (WASP). Florianópolis: Sociedade Brasileira de Computação, 2006. p. 1-10. 
     \item GOMES, LUIZ ; CÔRTES, M.L. . Comércio Eletrônico: Uma Análise da Aplicabilidade de Modelos de Qualidade de Software. In: Simpósio Brasileiro de Qualidade 
      de Software, 2002, Gramado. Anais do I Simpósio Brasileiro de Qualidade de Software, 2002. p. 1-2. 
    \item GOMES, LUIZ ; CÔRTES, M.L. . Suitability of Software Quality Models to e-Commerce Applications. In: Proceedings of the IADIS International Conference 
      WWW/Internet 2002, 2002, Lisboa. Proceedings of the IADIS International Conference WWW/Internet 2002, 2002. p. 1-2.
  \end{itemize}
 
  \section{IDIOMAS} 
  {\sl Inglês:} Compreendo bem, falo razoavelmente, leio muito bem, escrevo bem. 
  \begin{itemize} \itemsep -2pt % Reduce space between items
    \item TOEFL Paper Based Test(score 513 of 677) \hfill 2010
    \item Test of Writing English(score 4 of 6) \hfill 2010
    \item TOEIC Test(score 665 of 990)\hfill 2010
  \end{itemize}
 
  \section{CERTIFICA-ÇÕES} 
  OMG Certified UML Professional Fundamental\hfill 2011 \\
  IBM Certified Solution Designer OOAD vUML2\hfill 2011 \\
  Zend Certified Engineer PHP 5.3, Zend Technologies Inc\hfill 2011 \\
  IBM Certified Solution Designer RUP v7.0\hfill 2010 \\
  Sun Certified Developer MySQL 5, Sun Microsystems\hfill 2010 \\
  
  \section{HABILIDADES \\ TÉCNICAS} 
  {\sl Linguagens:} C, C$\#$, FORTRAN, Java, Latex, Pascal, PHP, Prolog,  Ruby, SQL e UML. \\
  {\sl Métodos e Metodologias:} Análise e Projetos Orientados a Objetos e Processo Unificado de Software.\\
  {\sl Softwares:} ArgoUML, AstahUML, Eclipse, JUnit, Git, MySQL, Netbeans, SVN, Visual Studio, Ubuntu Linux e OS X.\\

  
\end{resume}
\end{document}
