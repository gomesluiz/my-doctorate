\documentclass[11pt]{letter}
\input{pacotes}
\usepackage{geometry}
 \geometry{
  a4paper,
  total={210mm,297mm},
  left=30mm,
  right=20mm,
  top=30mm,
  bottom=20mm,
 }
\pagenumbering{gobble}

\begin{document}

\begin{flushleft}
\textbf{Instituto de Ciência da Computação}\\
Universidade Estadual de Campinas\\ 
Av. Albert Einstein, 1251\\
Cidade Universitária\\
13083.852 - Campinas, SP. 
\end{flushleft}


Prezada Comissão de Pós-graduação,

Escrevo esta carta com intuito de me apresentar como candidato a aluno regular do programa de Doutorado em Ciência da Computação deste Instituto desta Universidade. Meu nome é Luiz Alberto Ferreira Gomes, tenho 43 anos, e, atualmente, trabalho como professor do ensino 
superior no curso de Ciência da Computação da Pontifícia Universidade Católica de Minas Gerais (PUC Minas) no \textit{campus} da cidade de Poços de Caldas (localizada à 
170 km de Campinas). Nas próximas linhas, redigi minha trajetória acadêmica e profissional, destacando dela os principais eventos.

Estou envolvido com a área de Computação há mais de 20 anos, sendo que o primeiro contato com os seus conceitos aconteceu no início de 1988, quando ingressei no 
curso de Informática Industrial da Escola Técnica Federal de Ouro Preto. No primeiro semestre de 1993, após um breve período trabalhando em empresas da região de 
Ouro Preto (MG), prossegui os estudos ingressando no curso de Ciência da Computação da Universidade Federal de Ouro Preto (UFOP). Durante o curso, concomitantemente 
com as atividades acadêmicas, atuei como programador de computadores do Centro de Processamento de Dados da Universidade, sendo responsável por desenvolver diversos 
sistemas computacionais importantes para a instituição naquele momento. Com a obtenção do grau de bacharel no início de 1997 – e antes de ingressar em um programa 
de mestrado – tive a oportunidade de participar de alguns projetos de desenvolvimento de software de missão crítica para grandes empresas nacionais: Banco da Amazônia; 
Banco do Estado de Minas Gerais; Banco Mercantil do Brasil; Telemig Celular e Usiminas (em parceria com a Oracle do Brasil).

Ao término de um período de quatro anos, desempenhando o papel de analista de sistemas nos projetos de software para as empresas mencionadas acima, ingressei, no 
início de 2001, no programa de mestrado em Ciência da Computação da Universidade Estadual de Campinas (UNICAMP). O trabalho de pesquisa – orientado como muita 
competência pelo Prof. PhD. Mario Lúcio Côrtes – concentrou-se na área de Engenharia de Software, abordando questões relacionadas à aplicabilidade de modelos de 
qualidade de software em aplicações de comércio eletrônico e resultou, além do documento da dissertação, em dois artigos científicos: um publicado no 
Simpósio Brasileiro de Engenharia de Software e outro publicado na IADIS \textit{International Conference}. A obtenção do título de mestre, em abril de 2003, possibilitou 
que eu ocupasse de forma definitiva, após a aprovação em concurso, o cargo de Professor Assistente da PUC Minas.

Por último, gostaria de esclarecer que uma das razões que motivaram a candidatura ao programa Doutorado em Ciência da Computação 
se baseia na firme crença de empreender um projeto pesquisa sistematizado, consistente e robusto na área de manutenção de software, 
que, além de  melhorar as minhas habilidades como docente e pesquisador, origine diversas publicações e patentes relevantes; contribuindo, 
assim, com os esforços de pesquisa deste instituto e com o avanço do conhecimento nessa área.

\begin{flushright}
 
Grato,\\
Luiz Alberto Ferreira Gomes 
\end{flushright}


\end{document}
