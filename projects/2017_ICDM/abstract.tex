\begin{abstract}
In the context of Change Request (CR) systems, the severity level of a change request is considered a critical variable when planning software maintenance activities, indicating how soon a CR needs to be addressed. However, the severity level assignment remains primarily a manual process, mostly depending on the experience and expertise of the person who has reported the CR. This paper presents preliminary findings on the prediction of CR severity level by analyzing its long description, using text mining techniques and Machine Learning (ML) algorithms. We have collected CRs from three FLOSS projects (imbalanced) repositories: Cassandra, Hadoop and Spark. Ours results were better than those published in the literature in terms of F-measure performance for two research questions (using Random Forest) and similar for the third research question. However, subsequent analyses based on the Friedman test have demonstrated that data used in experiments haven't permitted us to say with enough confidence level that Random Forest is better than the others ML algorithms. We have also shown that the use classical ML measurements available in the literature may not help deciding whether a ML approach will bring any benefit to the user, and have proposed an alternative measuring approach to address this issue.
\end{abstract}

