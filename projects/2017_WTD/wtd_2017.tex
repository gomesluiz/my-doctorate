
%% bare_conf.tex
%% V1.3
%% 2007/01/11
%% by Michael Shell
%% See:
%% http://www.michaelshell.org/
%% for current contact information.
%%
%% This is a skeleton file demonstrating the use of IEEEtran.cls
%% (requires IEEEtran.cls version 1.7 or later) with an IEEE conference paper.
%%
%% Support sites:
%% http://www.michaelshell.org/tex/ieeetran/
%% http://www.ctan.org/tex-archive/macros/latex/contrib/IEEEtran/
%% and
%% http://www.ieee.org/

\documentclass[10pt, conference]{IEEEtran}

\usepackage{booktabs}
\usepackage{cite}
\usepackage{color}
\usepackage{enumitem}
\usepackage{float}
\usepackage[utf8]{inputenc}
\usepackage{tikz}
\usepackage{multicol}
\usepackage{multirow}
\usepackage{subcaption}
\usepackage[skip=0pt]{caption}
\usepackage{url}
\usepackage{rotating}

\makeatletter
\setlength{\@fptop}{0pt}
\makeatother

\newif\ifComments

% To turn comments OFF simply comment out the \Commentstrue line
\Commentstrue

\ifComments
\newcommand{\luiz}[1]{\noindent\textcolor{orange}{LUIZ: {#1}}}
\newcommand{\mario}[1]{\noindent\textcolor{green}{MARIO: {#1}}}
\newcommand{\rem}[1]{\noindent\textcolor{magenta}{REMOVED: {#1}}}
\newcommand{\new}[1]{\noindent\textcolor{blue}{NEW: {#1}}}
\else
\newcommand{\luiz}[1]{}
\newcommand{\mario}[1]{}
\newcommand{\rem}[1]{}
\newcommand{\new}[1]{#1}
\fi

% correct bad hyphenation here
\hyphenation{}

\begin{document}

\title{Machine Learning Based Prediction of CR Severity Level in FLOSS: Experimental Results
}

\author{
	\IEEEauthorblockN{Luiz Alberto Ferreira Gomes\\\texttt{gomes.luiz@ic.unicamp.br}}
	\IEEEauthorblockA{Institute of Computing, University of Campinas, Brazil.}
	\and
	\IEEEauthorblockN{Mario Lucio Cortes\\\texttt{cortes@ic.unicamp.br}}
	\IEEEauthorblockA{Institute of Computing, University of Campinas, Brazil}
}
\maketitle
\begin{abstract}
In the context of Change Request (CR) systems, the severity level of a change request is considered a critical variable when planning software maintenance activities, indicating how soon a CR needs to be addressed. However, the severity level assignment remains primarily a manual process, mostly depending on the experience and expertise of the person who has reported the CR. This paper presents preliminary findings on the prediction of CR severity level by analyzing its long description, using text mining techniques and Machine Learning (ML) algorithms. We have collected CRs from three FLOSS projects (imbalanced) repositories: Cassandra, Hadoop and Spark. Ours results were better than those published in the literature in terms of F-measure performance for two research questions (using Random Forest) and similar for the third research question. However, subsequent analyses based on the Friedman test have demonstrated that data used in experiments haven't permitted us to say with enough confidence level that Random Forest is better than the others ML algorithms. We have also shown that the use classical ML measurements available in the literature may not help deciding whether a ML approach will bring any benefit to the user, and have proposed an alternative measuring approach to address this issue.
\end{abstract}


\begin{IEEEkeywords}
software maintenance; change request systems; text mining; machine learning; software repositories.
\end{IEEEkeywords}

\IEEEpeerreviewmaketitle
\section{Introduction}\label{sec:introduction}

Bug Tracking Systems (BTS) have been playing a key role as a communication and collaboration tool in Closed Source Software (CSS) and Free/Libre Open Source Software (FLOSS). In both development environments, planning of software evolution and maintenance activities relies primarily on information of bug reports registered in this kind of system. This is particularly true in FLOSS, which is characterized by the existence of many users and developers with different levels of expertise spread out around the world, who might create or be responsible for dealing with several bug reports\cite{Cavalcanti:2014}. 

A user interacts with a BTS often through a simple mechanism called bug report form, which enables to communicate a bug to those in charge of developing or maintaining the software system\cite{Sommerville:2010}. Initially, he or she should inform a short description, a long description, and an associated severity level (e.g., blocker, critical, major, minor, and trivial). Subsequently, a software team member reviews this bug report and confirms or declines it (e.g., due to bug report duplication). If the bug report is confirmed, the team member should provide more information to complement the bug report form, for example, by indicating its priority and by assigning a person who will be responsible for fixing the bug. 

The information of severity level is recognized as a critical variable for prioritizing and planing how  bug reports will be dealt with\cite{Tian:2012}. It measures the impact the bug has on the successful execution of the software system and defines how soon the bug needs to be addressed\cite{Lamkanfi:2010}. However, the severity level assignment remains mostly a manual process, which relies only on the experience and expertise of the person who has opened the bug report\cite{Cavalcanti:2014, Tian:2012, Lamkanfi:2010}.  

Moreover, the number of bug reports in large and medium software FLOSS projects is frequently very large\cite{Yang:2017}. For example, Eclipse project had 84,245 bug reports opened from 2013 to 2015 alone, whereas Android project had over 107,456, and JBoss project had over 81,920. Therefore, a manual assignment of severity level may be a quite subjective, cumbersome and error-prone process, and a wrong decision throughout bug report lifecycle may strongly disturb the planning of maintenance activities. For instance, an important maintenance team resource could be allocated to address less significant bug reports before the most important ones.

Due to the evident importance of bug report severity information for the planning of FLOSS maintenance, both business and academic community have demonstrated great interest, and plenty of research has been done in this field. However, there are few mapping reviews well characterizing this area, positioning existing works and measuring current progress. Although these existing works helped to understand the challenges and opportunities in this research area \cite{Cavalcanti:2014, Uddin:2017}, they suffer from one shortcoming:  existing reviews lack in-depth coverage of the different dimensions of bug report severity prediction. Our current systematic mapping review fills this gap, by investigating and analyzing relevant papers \textemdash~published from 2010 to 2017~\textemdash~related to bug report severity prediction. To the best of our knowledge, this is the first review to provide a detailed document with the analysis of more than ten relevant aspects of the surveyed experiments, including: granularity of output class, FLOSS repositories, features and feature selection methods, text mining methods, machine learning algorithms, performance evaluation measures, sampling techniques, statistical tests,  experimental tools, and type of solution. 

A systematic mapping review aims to characterize the state-of-the-art on bug report severity prediction. This sort of study provides a broad overview of a research area to determine whether there is research evidence on a particular topic. According to Kitchenham et al.\cite{Kitchenham:2007} and Petersen et al.\cite{Petersen:2008}, results yielded by a mapping review help recognizing gaps to suggest future research and provide a direction to engage in new research activities appropriately. The key contributions of our review are three-fold: 

\begin{itemize}
  \item It proposes a categorization scheme to organize the experiments reported in papers on bug report severity prediction. For example, it provides a taxonomy to categorize the severity prediction problem type based on the granularity severity level which one wishes to predict.
  \item It provides an overview of the state-of-the-art research by summarizing patterns, procedures, and methods employed by researches to predict bug report severity. For example, it indicates which are the most used machine learning algorithms in reported experiments.
  \item It discusses challenges, and open issues and future directions in this research field to share the vision and expand the horizon of bug report severity prediction. 
\end{itemize}

The remaining of this paper is organized as follows: Section \ref{sec:background} provides the basic information background necessary to understand the research area. Section \ref{sec:relatedwork} presents related work. Section \ref{sec:research_method} describes our research method. Section \ref{sec:results} presents our results. Section \ref{sec:discussion} presents final findings and discussion. Finally, Section \ref{sec:conclusions} concludes the paper.
\section{Related Work}\label{sec:relatedwork} 

To the extent of our knowledge, only two papers\cite{Cavalcanti:2014, Uddin:2017} have reviewed the literature about bug report severity prediction. Cavalcanti et al.\cite{Cavalcanti:2014} performed a review of 142 papers, published between 2000 to 2012, that investigated challenges and opportunities for software change repositories. Just seven of them are related to change request prioritization, which is defined in Bugzilla by two fields: \textit{priority} and \textit{severity}. Four out of them addressed bug report severity prediction. Only two papers (Lamkanfi et al.\cite{Lamkanfi:2010} and Lamkanfi et al.\cite{Lamkanfi:2011}), however, addressed severity prediction on FLOSS projects. That mapping review shows that all of seven papers used some Information Retrieval Model: one paper used the vector space representation (binary and term count), and all seven used TF-IDF. It also shows that all papers implement learning techniques such as SVM (4 out of 7), Decision tree (2 out of 7), k-NN (2 out of 7), Naïve Bayes (3 out of 7), and Naïve Bayes Multinomial (1 out of 7).

The review presented by Uddin et al. \cite{Uddin:2017}, in turn, surveyed published papers in bug prioritization. The authors reviewed and analyzed in depth 32 distinct papers published between 2003 and 2015. The aim of that analysis was ``to summarize the existing work on bug prioritization and some problems in working with bug prioritization". That work categorizes research initiatives according to ML algorithms, evaluation measures, data sets, researchers, and publication venue. It can be worth to note that the authors investigated only eight papers about predicting of severity level. In contrast, the current mapping review investigated in depth 27 papers about bug report severity prediction published from 2010 to 2017.

Although these reviews present relevant results for research in this area, they have one difference to our work. There is no explicit focus on bug report severity prediction. In fact, those papers perform a brief review of this topic by addressing mainly bug report priority prediction. The current review has a broader goal of mapping studies addressing many research questions and concepts related to bug report severity prediction. Furthermore, this review considers relevant aspects (e.g., sampling techniques and statistical tests), not addressed before, in the characterization of papers.
\section{BACKGROUND} \label{sec:background} 
 

This section briefly comments of basic concepts necessary to understand this research area, namely CR Systems, Text Mining, Machine Learning, and ML evaluation metrics.

Data used in this research area are usually extracted from the so-called CR Systems, or Bug Tracking Systems. Popular CR Systems are Bugzilla, Jira, and Redmine \cite{Tian2012}. In this work, we extracted Cassandra, HADOOP and Spark datasets from Jira CR System. Additional information can be found in \cite{Pressman2009}.

Two techniques are frequently used in this research area: Text Mining  \cite{Feldman2007} \cite{Williams2011} and Machine Learning (ML) \cite{Williams2011} \cite{Surya2016} \cite{Russell2010} \cite{Breiman2001}. Detailing of these techniques are outside the scope of this paper.

Finally, it is worth mentioning the specific metrics we use for assessing prediction performance. The three most common performance measures for evaluating the accuracy of classification algorithms are precision, recall, and F-measure, described as follows \cite{Facelli2015} \cite{Zhao2013}:

\textbf{Recall}. Recall is the number of True Positives (TP) divided by the number of True Positives (TP) and of False Negatives (FN), where the TP and FN values are derived from the confusion matrix. A low recall indicates many false negatives.

\textbf{Precision}. Precision is the number of True Positives (TP) divided by the number of True Positives and False Positives (FP). A low precision can also indicate many false positives.

\textbf{F-measure}. F-measure conveys the balance between precision and recall, and can be calculated as their harmonic mean. 


\section{Experiment} \label{sec:experiment}
This section describes the experiment conducted to address the Research Questions. As in typical methodologies used in ML studies, it comprises the following steps: Data Collection (\ref{subsec:collection}), Data Preprocessing (Section \ref{subsec:preprocessing}), and Training and Testing (Section \ref{subsec:training}).

\subsection{Data Collection} 	\label{subsec:collection}
This step in the experimental research encompasses selecting FLOSS datasets to serve as the data source, studying and interpreting its data structure, and finally extracting relevant data from its repository (feature extraction). In this research, Cassandra, Hadoop, Linux, Mozilla, and Spark Open Systems were considered as potential Open Source Systems to study. In a first approximation, Cassandra, Hadoop and Spark were selected as data sources of CR records, due to the fact they are open, well stablished, have a considerable number of CRs already registered, use standard repositories, and were under study by other researchers in our research group.

Cassandra[cassandra.apache.org] is a distributed NoSQL database management system designed to handle large amounts of data across many servers, providing fault-tolerance with no single point of failure. Hadoop[hadoop.apache.org] is a framework that allows for the distributed processing of large data sets across clusters of computers using simple programming models. Spark[spark.apache.org] is a cluster-computing engine for large-scale data processing which provides an interface for programming entire clusters with implicit data parallelism and fault-tolerance. They are considered a specialized and complex FLOSS project with many users with different levels of expertise.

CRs from these FLOSS projects are stored in a Jira based repository [\url{https://www.atlassian.com/software/jira}] which allows for access to all CR contents in XML format. Everything is available (except change history), from CR long description field (with lines with few characters to ones with many lines), including code snippets and exception stack trace. Two steps are used to perform data extraction from their web site[\url{http://issues.apache.org}]: (i) copying CR basic data (e.g. status and resolution) from XML contents; and (ii) copying CR changes history from external HTML pages (this may be important for learning).

CR record data from February 01, 2006 to May 07, 2017 were collected.  The total number of CR records retrieved after preprocessing was 22901. 
 
\begin{figure*}[!ht]
  \centering
  \includegraphics[scale=0.7]{figures/issues_distribution_cassandra.pdf}    
  \caption{Cassandra dataset distributions: (a) severity level (b) change pattern (c) direction of change.}
  \label{fig:issues_distribution_cassandra}
\end{figure*}

\begin{figure*}[!ht]
  \centering
  \includegraphics[scale=0.7]{figures/issues_distribution_hadoop.pdf}
  \caption{Hadoop dataset distributions: (a) severity level (b) change pattern (c) direction of change.}
  \label{fig:issues_distribution_hadoop}
\end{figure*}

\begin{figure*}[!ht]
  \centering
  \includegraphics[scale=0.7]{figures/issues_distribution_spark.pdf}   
  \caption{Spark dataset distributions: (a) severity level (b) change pattern (c) direction of change.}
  \label{fig:issues_distribution_spark}
\end{figure*}


Figure \ref{fig:issues_distribution_cassandra} shows how the 7538 retrieved CR Cassandra records were distributed in terms of severity level and severity level change. Figure \ref{fig:issues_distribution_cassandra}(a) shows the severity level distribution: 9.7\% have severity trivial (1); 37.6\% have severity minor (2); 48.4\% have severity major (3), 3.0\% have severity critical (4), and 1.3\% have severity blocker (5). Figure \ref{fig:issues_distribution_cassandra}(b) shows that only 7.0\% have changed their severities levels during the CR lifecycle. Finally, Figure \ref{fig:issues_distribution_cassandra}(c) reveals that of these 7.0\% CRs which changed their severity, 67\% decreased it, and 33\% increased it. 

Figure \ref{fig:issues_distribution_hadoop} shows how the 8262 retrieved CR Hadoop records from were distributed in terms of severity level and severity level change. Figure \ref{fig:issues_distribution_hadoop}(a) shows the severity level distribution: 4.3\% have severity trivial (1); 19.6\% have severity minor (2); 61.2\% have severity major (3), 3.8\% have severity critical (4), and 11.1\% have severity blocker (5). Figure \ref{fig:issues_distribution_hadoop}(b) shows that only 8.0\% have changed their severities levels during the CR lifecycle. Finally, Figure \ref{fig:issues_distribution_hadoop}(c) reveals that of these 8.0\% CRs which changed their severity, 30.8\% decreased it, and 69.2\% increased it. 

Figure \ref{fig:issues_distribution_spark} shows how the 7101 retrieved CR Spark records were distributed in terms of severity level and severity level change. Figure \ref{fig:issues_distribution_spark}(a) shows the severity level distribution: 2.9\% have severity trivial (1); 4.4\% have severity minor (2); 22.5\% have severity major (3), 50.6\% have severity critical (4), and 10.1\% have severity blocker (5). Figure \ref{fig:issues_distribution_spark}(b) shows that only 13.3\% have changed their severities levels during the CR lifecycle. Finally, Figure \ref{fig:issues_distribution_spark}(c) reveals that of these 13.3\% CRs which changed their severity, 35.9\% decreased it, and 64.1\% increased it. 

Summarizing findings in Figures \ref{fig:issues_distribution_cassandra}, \ref{fig:issues_distribution_hadoop} and  \ref{fig:issues_distribution_spark}: (a) the most frequent severity level type is ``major''; (b) there have been few changes in severity levels; (c) in two of them (Hadoop and Spark), severity levels increased. Furthermore, one can see that these datasets are clearly imbalanced, posing additional difficulty to the application of the ML methodology. 

\subsection{Preprocessing} 	\label{subsec:preprocessing}

Raw data previously collected from the Cassandra, Hadoop and Spark CR repositories were not properly structured to serve as input to ML algorithms, they weren't in tidy data format \cite{DeJonge2013}. The classical way to address this problem is to run preprocessing procedures to extract, organize and structure relevant features out of the raw data. Specific scripts were written in R language to accomplish this features extraction. Preprocessing tasks were executed as follows:  

\begin{itemize}
 \item Extraction of relevant features: key, type, status, resolution status, days to resolve, quantity of comments, severity level and long description of CRs; 
 \item Selecting only CRs with status equals to Closed and resolution equals to Fixed and Implemented. This type of CRs was effectively implemented by the development team and they can no longer have their severity level changed.
 \item Merging CR features with their change history data. This additional information allows for the identification of CRs that have changed severity level during the CR lifecycle, and furthermore, if they have changed for better (decrease) or worse (increase).
 \item Performing text mining in the long description field to identify the 100 most frequent words. This information is then converted into features for each CR.
\end{itemize}

\subsection{Training and testing}  	\label{subsec:training}

Training and testing steps start with partitioning the already preprocessed dataset in two disjoint subsets: a subset for training, with 80\% of the CRs, and a subset for testing, with the remaining 20\% of the CRs. Three classical sampling approaches, random, proportional, and uniform\cite{Japkowicz:2011} were analyzed to select the training set. Best results were obtained with the random sampling technique. In the training phase, we have used the 5$\times$3 Repeated Cross-Validation technique \cite{Zhao2013} to obtain more stable estimates of each algorithm's performance and enhance replicability of the results \cite{Japkowicz:2011}. In the testing phase, each ML algorithm was validated with 20\% of each CR dataset to measure its accuracy.
 
We have chosen three traditional ML algorithms: Neural Networks\cite{Haykin:1998}, Random Forest\cite{Breiman2001} and Support Vector Machine(SVM)\cite{Cristianini:1999} which were implemented, respectively, using neuralnet (with Single Hidden Layer), randomForest, and kernlab (with Radial Basis Function Kernel and multi-class classification) R libraries.




\section{Findings and Discussions}  \label{sec:discussion}
This section presents the experimental findings listed by research question: RQ1 (\ref{subsec:rq1}), RQ2 (Section \ref{subsec:rq2}), RQ3 (Section \ref{subsec:rq3}) and RQ4 (Section \ref{subsec:rq4}). In addition, the performance of ML algorithms is evaluated using Friedman Statistical Test (Section \ref{subsec:tests}). 

\subsection{RQ1: Will the CR severity level change?}\label{subsec:rq1}

The RQ1 is a simple binary problem, i.e., a question whose answer is true (class 1) or false (class 0). Tables \ref{tab:classifiers_precision_on_rq1}, \ref{tab:classifiers_recall_on_rq1} and \ref{tab:classifiers_f_measure_on_rq1} 
show the performance of ML algorithms to predict the response to this issue.

\begin{table}[!ht]
	\renewcommand{\arraystretch}{1.8}
	\caption{ML algorithms precision performance on RQ1.}
	\label{tab:classifiers_precision_on_rq1}
	\centering
	\begin{tabular}{l|c|c|c|c|}
		\cline{2-5}
		& Class & Neural Network & Random Forest & SVM\\
		\cline{1-5}
		\multicolumn{5}{ |c| }{Precision}\\
		\cline{1-5} 
        \multicolumn{1}{ |c| }{\multirow{2}{*}{\rotatebox[origin=c]{90}{\scriptsize{Cassandra}}}} & 0 & 0.9530591 & 0.9596013 & 0.9581371\\
		\cline{2-5}
		\multicolumn{1}{ |c| }{} & 1 & 0.7241379 & 0.9740260 & 1.0000000\\
		\hline
		\multicolumn{1}{ |c| }{\multirow{2}{*}{\rotatebox[origin=c]{90}{\scriptsize{Hadoop}}}} & 0 & 0.9530516 & 0.9498208 & 0.9477157\\
		\cline{2-5}
		\multicolumn{1}{ |c| }{} & 1 & 0.6339286 & 0.8289474 & 0.9830508\\
		\hline
		\multicolumn{1}{ |c| }{\multirow{2}{*}{\rotatebox[origin=c]{90}{\scriptsize{Spark}}}} 		
		& 0 & 0.9125249 & 0.9274406 & 0.9134555\\
		\cline{2-5}
		\multicolumn{1}{ |c| }{} & 1 & 0.6162162 & 0.7640449 & 0.9819820\\
		\cline{1-5} 
		\multicolumn{1}{ |c| }{} & Average & 0.7988197 & 0.9006468 & 0.9640569\\
		\hline 
	\end{tabular}
\end{table}
%
\begin{table}[!ht]
	\renewcommand{\arraystretch}{1.8}
	\caption{ML algorithms recall performance on RQ1.}
	\label{tab:classifiers_recall_on_rq1}
	\centering
	\begin{tabular}{l|c|c|c|c|}
		\cline{2-5}
		& Class & Neural Network & Random Forest & SVM\\
		\cline{1-5}		
		\multicolumn{5}{ |c| }{Recall}\\
		\cline{1-5} 
        \multicolumn{1}{ |c| }{\multirow{2}{*}{\rotatebox[origin=c]{90}{\scriptsize{Cassandra}}}} & 0 & 0.9868924 & 0.9989077 & 1.0000000\\
		\cline{2-5}
		\multicolumn{1}{ |c| }{} & 1 & 0.4144737 & 0.4934211 & 0.4736842\\
		\hline
		\multicolumn{1}{ |c| }{\multirow{2}{*}{\rotatebox[origin=c]{90}{\scriptsize{Hadoop}}}} & 0 & 0.9780514 & 0.9930407 & 0.9994647\\
		\cline{2-5}
		\multicolumn{1}{ |c| }{} & 1 & 0.4409938 & 0.3913043 & 0.3602484\\
		\hline
		\multicolumn{1}{ |c| }{\multirow{2}{*}{\rotatebox[origin=c]{90}{\scriptsize{Spark}}}} 		
		& 0 & 0.9509669 & 0.9709945 & 0.9986188\\
		\cline{2-5}
		\multicolumn{1}{ |c| }{} & 1 & 0.4634146 & 0.5528455 & 0.4430894\\
		\cline{1-5} 
		\multicolumn{1}{ |c| }{} & Average & 0.7057988 & 0.7334190 & 0.7125176\\
		\hline 
	\end{tabular}
\end{table}
%
\begin{table}[!ht]
	\renewcommand{\arraystretch}{1.8}
	\caption{ML algorithms F-measure performance on RQ1.}
	\label{tab:classifiers_f_measure_on_rq1}
	\centering
	\begin{tabular}{l|c|c|c|c|}
		\cline{2-5}
		& Class & Neural Network & Random Forest & SVM\\
		\cline{1-5}		
		\multicolumn{5}{ |c| }{F-measure}\\
		\cline{1-5} 
        \multicolumn{1}{ |c| }{\multirow{2}{*}{\rotatebox[origin=c]{90}{\scriptsize{Cassandra}}}} & 0 & 0.9696807 & 0.9788600 & 0.9786211\\
		\cline{2-5}
		\multicolumn{1}{ |c| }{} & 1 & 0.5271967 & 0.6550218 & 0.6428571\\
		\hline
		\multicolumn{1}{ |c| }{\multirow{2}{*}{\rotatebox[origin=c]{90}{\scriptsize{Hadoop}}}} & 0 & 0.9653897 & 0.9709500 & 0.9729026\\
		\cline{2-5}
		\multicolumn{1}{ |c| }{} & 1 & 0.5201465 & 0.5316456 & 0.5272727\\
		\hline
		\multicolumn{1}{ |c| }{\multirow{2}{*}{\rotatebox[origin=c]{90}{\scriptsize{Spark}}}} 		
		& 0 & 0.9313493 & 0.9487179 & 0.9541405\\
		\cline{2-5}
		\multicolumn{1}{ |c| }{} & 1 & 0.5290023 & 0.6415094 & 0.6106443\\
		\cline{1-5} 
		\multicolumn{1}{ |c| }{} & Average & 0.7404609 & 0.7877841 & 0.7810730\\
		\hline 
	\end{tabular}
\end{table}

We tested the ML algorithms with 4580 (20\% of 22901) CRs: 4154 have changed their severity level, and 426 haven't changed their severity level. We can observe that the three algorithms performed very closely. However, the Random Forest algorithm had the best performance in two out of three measures. 

We have also analyzed the percentage of incorrect predictions made in answering RQ1. Figure \ref{fig:classifiers_performance_on_q1q2q3} shows that this implementation of the Neural Network Algorithm had the worst performance and the two others performed similarly. 

\begin{figure*}[!ht]
  \centering
  \includegraphics[scale=1]{figures/classifiers_performance_on_q1q2q3.pdf}
  \caption{Performance of ML algorithms for RQ1 and RQ2.}
  \label{fig:classifiers_performance_on_q1q2q3}
\end{figure*}

\subsection{RQ2: Will the CR severity level increase, decrease or remain the same?}\label{subsec:rq2}

The RQ2 poses a problem more difficult than the previous question. It is a question with three possible responses related to severity level: it has decreased (class -1); it has remained (class 0); and it has increased (class 1). Tables \ref{tab:classifiers_precision_on_rq2}, \ref{tab:classifiers_recall_on_rq2}, and \ref{tab:classifiers_f_measure_on_rq2} shows the performance of the ML algorithms to predict the response to this issue. 

\begin{table}[!ht]
	\renewcommand{\arraystretch}{1.8}
	\caption{ML algorithms precision performance on RQ2.}
	\label{tab:classifiers_precision_on_rq2}
	\centering
	\begin{tabular}{l|c|c|c|c|}
		\cline{2-5}
		& Class & Neural Network & Random Forest & SVM\\
		\cline{1-5}	
		\multicolumn{5}{ |c| }{Precision}\\
		\cline{1-5} 
        \multicolumn{1}{ |c| }{\multirow{3}{*}{\rotatebox[origin=c]{90}{\scriptsize{Cassandra}}}} & -1 & 0.4393939 & 0.7435897 & 0.7631579\\
		\cline{2-5}
		\multicolumn{1}{ |c| }{} & 0 & 0.9523305 & 0.9565445 & 0.9546403\\
		\cline{2-5}
		\multicolumn{1}{ |c| }{} & 1 & 0.7333333 & 0.7428571 & 0.8214286\\
		\hline
		\multicolumn{1}{ |c| }{\multirow{3}{*}{\rotatebox[origin=c]{90}{\scriptsize{Hadoop}}}} & -1 & 0.2272727 & 0.5555556 & 0.5000000\\
		\cline{2-5}
		\multicolumn{1}{ |c| }{} & 0 & 0.9549266 & 0.9551084 & 0.9520653\\
		\cline{2-5}
		\multicolumn{1}{ |c| }{} & 1 & 0.6060606 & 0.7195122 & 0.8666667\\
        \hline
		\multicolumn{1}{ |c| }{\multirow{2}{*}{\rotatebox[origin=c]{90}{\scriptsize{Spark}}}} 	& -1 & 0.2500000 & 0.5925926 & 0.5769231\\
		\cline{2-5}
		\multicolumn{1}{ |c| }{} & 0 & 0.9119788 & 0.9303548 & 0.9157695\\
		\cline{2-5}
		\multicolumn{1}{ |c| }{} & 1 & 0.5555556 & 0.6689655 & 0.7977528\\
		\cline{1-5} 
		\multicolumn{1}{ |c| }{} & Average & 0.6256502 & 0.7627867 & 0.7942671\\
		\hline
	\end{tabular}
\end{table}

\begin{table}[!ht]
	\renewcommand{\arraystretch}{1.8}
	\caption{ML algorithms recall performance on RQ2.}
	\label{tab:classifiers_recall_on_rq2}
	\centering
	\begin{tabular}{l|c|c|c|c|}
		\cline{2-5}
		& Class & Neural Network & Random Forest & SVM\\
		\cline{1-5}	
		\multicolumn{5}{ |c| }{Recall}\\
		\cline{1-5} 
        \multicolumn{1}{ |c| }{\multirow{3}{*}{\rotatebox[origin=c]{90}{\scriptsize{Cassandra}}}} & -1 & 0.2989691 & 0.2989691 & 0.29896907\\
		\cline{2-5}
		\multicolumn{1}{ |c| }{} & 0 & 0.9819771 & 0.9978154 & 1.00000000\\
		\cline{2-5}
		\multicolumn{1}{ |c| }{} & 1 & 0.3928571 & 0.4642857 & 0.41071429\\
		\hline
		\multicolumn{1}{ |c| }{\multirow{3}{*}{\rotatebox[origin=c]{90}{\scriptsize{Hadoop}}}} & -1 & 0.1111111 & 0.1111111 & 0.08888889\\
		\cline{2-5}
		\multicolumn{1}{ |c| }{} & 0 & 0.9753747 & 0.9908994 & 0.99946467\\
		\cline{2-5}
		\multicolumn{1}{ |c| }{} & 1 & 0.5172414 & 0.5086207 & 0.44827586\\
        \hline
		\multicolumn{1}{ |c| }{\multirow{2}{*}{\rotatebox[origin=c]{90}{\scriptsize{Spark}}}} 	& -1 & 0.1500000 & 0.2000000 & 0.18750000\\
		\cline{2-5}
		\multicolumn{1}{ |c| }{} & 0 & 0.9516575 & 0.9779006 & 0.99861878\\
		\cline{2-5}
		\multicolumn{1}{ |c| }{} & 1 & 0.4518072 & 0.5843373 & 0.42771084\\
		\cline{1-5} 
		\multicolumn{1}{ |c| }{} & Average & 0.5367772 & 0.5704376 & 0.5400158\\
		\hline
	\end{tabular}
\end{table}

\begin{table}[!ht]
	\renewcommand{\arraystretch}{1.8}
	\caption{ML algorithms F-measure performance on RQ2.}
	\label{tab:classifiers_f_measure_on_rq2}
	\centering
	\begin{tabular}{l|c|c|c|c|}
		\cline{2-5}
		& Class & Neural Network & Random Forest & SVM\\
		\cline{1-5}	
		\multicolumn{5}{ |c| }{F-measure}\\
		\cline{1-5} 
        \multicolumn{1}{ |c| }{\multirow{3}{*}{\rotatebox[origin=c]{90}{\scriptsize{Cassandra}}}} & -1 & 0.3558282 & 0.4264706 & 0.4296296\\
		\cline{2-5}
		\multicolumn{1}{ |c| }{} & 0 & 0.9669266 & 0.9767442 & 0.9767938\\
		\cline{2-5}
		\multicolumn{1}{ |c| }{} & 1 & 0.5116279 & 0.5714286 & 0.5476190\\
		\hline
		\multicolumn{1}{ |c| }{\multirow{3}{*}{\rotatebox[origin=c]{90}{\scriptsize{Hadoop}}}} & -1 & 0.1492537 & 0.1851852 & 0.1509434\\
		\cline{2-5}
		\multicolumn{1}{ |c| }{} & 0 & 0.9650424 & 0.9726747 & 0.9751893\\
		\cline{2-5}
		\multicolumn{1}{ |c| }{} & 1 & 0.5581395 & 0.5959596 & 0.5909091\\
        \hline
		\multicolumn{1}{ |c| }{\multirow{2}{*}{\rotatebox[origin=c]{90}{\scriptsize{Spark}}}} 	& -1 & 0.1875000 & 0.2990654 & 0.2830189\\
		\cline{2-5}
		\multicolumn{1}{ |c| }{} & 0 & 0.9313957 & 0.9535354 & 0.5290023\\
		\cline{2-5}
		\multicolumn{1}{ |c| }{} & 1 & 0.4983389 & 0.6237942 & 0.5568627\\
		\cline{1-5} 
		\multicolumn{1}{ |c| }{} & Average & 0.5693392 & 0.6227619 & 0.6073741\\
		\hline
	\end{tabular}
\end{table}

We tested the ML algorithms with 4580 (20\% of 22901) CR. Only now, we have three predicting situations: 4154 haven't changed their severity level, 246 have increased their severity level, and 180 have decreased their severity level. We can observe in Tables \ref{tab:classifiers_precision_on_rq2}, \ref{tab:classifiers_recall_on_rq2} and  \ref{tab:classifiers_f_measure_on_rq2} which the ML algorithms also performed very closely as question 1. However, the Random Forest algorithm had also the best performance in two out of three measures.

Figure \ref{fig:classifiers_performance_on_q1q2q3} shows that the ML algorithms performance had the same pattern in RQ2, as compared to RQ1. 

\subsection{RQ3: What is the prediction for the final CR severity level?}\label{subsec:rq3}

The RQ3 is a problem much harder than other two. It is a question with five responses related to severity level: (1) trivial; (2) minor; (3) major; (4) critical; and (5) blocker. Tables \ref{tab:classifiers_precision_on_rq3}, \ref{tab:classifiers_precision_on_rq3} and \ref{tab:classifiers_precision_on_rq3} shows the performance of the ML algorithms to predict the response to this issue. 

\begin{table}[!ht]
	\renewcommand{\arraystretch}{1.8}
	\caption{ML algorithms precision performance on RQ3.}
	\label{tab:classifiers_precision_on_rq3}
	\centering
	\begin{tabular}{l|c|c|c|c|}
		\cline{2-5}
		& Class & Neural Network & Random Forest & SVM\\
		\cline{1-5}	
		\multicolumn{5}{ |c| }{Precision}\\
		\cline{1-5} 
        \multicolumn{1}{ |c| }{\multirow{5}{*}{\rotatebox[origin=c]{90}{\scriptsize{Cassandra}}}} & 1 & 0.4022989 & 0.6976744 & 0.5348837\\
		\cline{2-5}
		\multicolumn{1}{ |c| }{} & 2 & 0.5394737 & 0.6209440 & 0.7513966\\
		\cline{2-5}
		\multicolumn{1}{ |c| }{} & 3 & 0.6645221 & 0.6924959 & 0.6222510\\
		\cline{2-5}
		\multicolumn{1}{ |c| }{} & 4 & 0.4782609 & 1.0000000 & 1.0000000\\
		\cline{2-5}
		\multicolumn{1}{ |c| }{} & 5 & 0.6666667 & 1.0000000 & 1.0000000\\
		\hline
		\multicolumn{1}{ |c| }{\multirow{5}{*}{\rotatebox[origin=c]{90}{\scriptsize{Hadoop}}}} & 1 & 0.2000000 & 0.9333333 & 0.8750000\\
		\cline{2-5}
		\multicolumn{1}{ |c| }{} & 2 & 0.3964497 & 0.7433628 & 0.9452055\\
		\cline{2-5}
		\multicolumn{1}{ |c| }{} & 3 & 0.6668558 & 0.7057175 & 0.6960305\\
		\cline{2-5}
		\multicolumn{1}{ |c| }{} & 4 & 0.3333333 & 1.0000000 & 1.0000000\\
		\cline{2-5}
		\multicolumn{1}{ |c| }{} & 5 & 0.4032258 & 0.9204545 & 1.0000000\\
        \hline
		\multicolumn{1}{ |c| }{\multirow{5}{*}{\rotatebox[origin=c]{90}{\scriptsize{Spark}}}} 	& 1 & 0.1818182 & 0.7142857 & 0.8000000\\
		\cline{2-5}
		\multicolumn{1}{ |c| }{} & 2 & 0.3345070 & 0.4785276 & 0.8194444\\
		\cline{2-5}
		\multicolumn{1}{ |c| }{} & 3 & 0.6096892 & 0.6131657 & 0.5994532\\
		\cline{2-5}
		\multicolumn{1}{ |c| }{} & 4 & 0.4807692 & 0.9733333 & 0.9726027\\
		\cline{2-5}
		\multicolumn{1}{ |c| }{} & 5 & 0.3703704 & 0.7857143 & 0.9500000\\
		\cline{1-5} 
		\multicolumn{1}{ |c| }{} & Average & 0.4485493
 & 0.7919339 & 0.8377511 \\
		\hline
	\end{tabular}
\end{table}
%
\begin{table}[!ht]
	\renewcommand{\arraystretch}{1.8}
	\caption{ML algorithms recall performance on RQ3.}
	\label{tab:classifiers_recall_on_rq3}
	\centering
	\begin{tabular}{l|c|c|c|c|}
		\cline{2-5}
		
		& Class & Neural Network & Random Forest & SVM\\
		\cline{1-5}	
		\multicolumn{5}{ |c| }{Recall}\\
		\cline{1-5} 
        \multicolumn{1}{ |c| }{\multirow{5}{*}{\rotatebox[origin=c]{90}{\scriptsize{Cassandra}}}} & 1 & 0.2243589 & 0.1923077 & 0.1474359\\
		\cline{2-5}
		\multicolumn{1}{ |c| }{} & 2 & 0.5766526 & 0.5921238 & 0.3783404\\
		\cline{2-5}
		\multicolumn{1}{ |c| }{} & 3 & 0.7053658 & 0.8282927 & 0.9385366\\
		\cline{2-5}
		\multicolumn{1}{ |c| }{} & 4 & 0.3283582 & 0.4626866 & 0.4626866\\
		\cline{2-5}
		\multicolumn{1}{ |c| }{} & 5 & 0.0800000 & 0.2400000 & 0.2400000\\
		\hline
		\multicolumn{1}{ |c| }{\multirow{5}{*}{\rotatebox[origin=c]{90}{\scriptsize{Hadoop}}}} & 1 & 0.0131578 & 0.1842105 & 0.1842105\\
		\cline{2-5}
		\multicolumn{1}{ |c| }{} & 2 & 0.1850828 & 0.2320442 & 0.1906077\\
		\cline{2-5}
		\multicolumn{1}{ |c| }{} & 3 & 0.9136858 & 0.9790047 & 0.9953344\\
		\cline{2-5}
		\multicolumn{1}{ |c| }{} & 4 & 0.1265822 & 0.3544304 & 0.3417722\\
		\cline{2-5}
		\multicolumn{1}{ |c| }{} & 5 & 0.1111111 & 0.3600000 & 0.3244444\\
        \hline
		\multicolumn{1}{ |c| }{\multirow{5}{*}{\rotatebox[origin=c]{90}{\scriptsize{Spark}}}} 	& 1 & 0.0312500 & 0.0781250 & 0.0625000\\
		\cline{2-5}
		\multicolumn{1}{ |c| }{} & 2 & 0.2691218 & 0.2209632 & 0.1671388\\
		\cline{2-5}
		\multicolumn{1}{ |c| }{} & 3 & 0.7494382 & 0.9314607 & 0.9853933\\
		\cline{2-5}
		\multicolumn{1}{ |c| }{} & 4 & 0.3989361 & 0.3882979 & 0.3776596\\
		\cline{2-5}
		\multicolumn{1}{ |c| }{} & 5 & 0.2531645 & 0.2784810 & 0.2405063\\
		\cline{1-5} 
		\multicolumn{1}{ |c| }{} & Average & 0.3310844
 & 0.4214952 & 0.4024377 \\
		\hline
	\end{tabular}
\end{table}
%
\begin{table}[!ht]
	\renewcommand{\arraystretch}{1.8}
	\caption{ML algorithms F-measure performance on RQ3.}
	\label{tab:classifiers_f_measure_on_rq3}
	\centering
	\begin{tabular}{l|c|c|c|c|}
		\cline{2-5}
		
		& Class & Neural Network & Random Forest & SVM\\
		\cline{1-5}	
		\multicolumn{5}{ |c| }{F-measure}\\
		\cline{1-5} 
        \multicolumn{1}{ |c| }{\multirow{5}{*}{\rotatebox[origin=c]{90}{\scriptsize{Cassandra}}}} & 1 & 0.2880658 & 0.3015075 & 0.2311558\\
		\cline{2-5}
		\multicolumn{1}{ |c| }{} & 2 & 0.5574439 & 0.6061915 & 0.5032741\\
		\cline{2-5}
		\multicolumn{1}{ |c| }{} & 3 & 0.6843350 & 0.7543314 & 0.7483469\\
		\cline{2-5}
		\multicolumn{1}{ |c| }{} & 4 & 0.3893805 & 0.6326531 & 0.6326531\\
		\cline{2-5}
		\multicolumn{1}{ |c| }{} & 5 & 0.1428571 & 0.3870968 & 0.3870968\\
		\hline
		\multicolumn{1}{ |c| }{\multirow{5}{*}{\rotatebox[origin=c]{90}{\scriptsize{Hadoop}}}} & 1 & 0.0246913 & 0.3076923 & 0.3043478\\
		\cline{2-5}
		\multicolumn{1}{ |c| }{} & 2 & 0.2523540 & 0.3536842 & 0.3172414\\
		\cline{2-5}
		\multicolumn{1}{ |c| }{} & 3 & 0.7709973 & 0.8201954 & 0.8192000\\
		\cline{2-5}
		\multicolumn{1}{ |c| }{} & 4 & 0.1834862 & 0.5233645 & 0.5094340\\
		\cline{2-5}
		\multicolumn{1}{ |c| }{} & 5 & 0.1742160 & 0.5175719 & 0.4899329\\
        \hline
		\multicolumn{1}{ |c| }{\multirow{5}{*}{\rotatebox[origin=c]{90}{\scriptsize{Spark}}}} 	& 1 & 0.0533333 & 0.1408451 & 0.1159420\\
		\cline{2-5}
		\multicolumn{1}{ |c| }{} & 2 & 0.2982731 & 0.3023256 & 0.2776471\\
		\cline{2-5}
		\multicolumn{1}{ |c| }{} & 3 & 0.6723790 & 0.7395183 & 0.7454314\\
		\cline{2-5}
		\multicolumn{1}{ |c| }{} & 4 & 0.4360465 & 0.5551331 & 0.5440613\\
		\cline{2-5}
		\multicolumn{1}{ |c| }{} & 5 & 0.3007518 & 0.4112150 & 0.3838384\\
		\cline{1-5} 
		\multicolumn{1}{ |c| }{} & Average & 0.3485740
 & 0.4902217 & 0.4673068 \\
		\hline
	\end{tabular}
\end{table}

We tested the ML algorithms with 4580 (20\% of 22901) CRs. Only now, we have six predicting situations: 288 are trivial; 1218 are minor; 2470 are major; 259 are critical; 345 are a blocker. We can observe in the Table \ref{tab:classifiers_precision_on_rq3}, \ref{tab:classifiers_recall_on_rq3} and \ref{tab:classifiers_f_measure_on_rq3} which the ML algorithms also performed very closely as questions 1 and 2. As in the two previous questions, the Random Forest algorithm had also the best performance in two out of three measures.

Figure \ref{fig:classifiers_performance_on_q1q2q3} shows that the ML algorithms performance had the same pattern in RQ1 and RQ2, as compared to RQ3.

\subsection{RQ4: How ML predictions compare to user prediction?}\label{subsec:rq4}
We have compared ML algorithms predictions to user prediction in terms of error magnitude. Figure \ref{fig:classifiers_performance_for_q3} shows predictors versus user error magnitude in the assignment of severity level.  Figure \ref{fig:classifiers_performance_for_q4} analyzes how well the ML prediction performed with respect to user prediction: better (ML algorithm error absolute value was smaller than user prediction error), equals (ML algorithm error equals to user error) or worse (ML algorithm error greater than user error).  The data clearly show that the use of this type of software predictor results in no gain to the user. This conclusion could not be drawn simply knowing the value of the classic accuracy measurement for Neural Network (2426/4580 = 52.969\%), Random Forest (2762/4580 = 60.305\%), and SVM (2731/4580 = 59.628\%) (see Figure \ref{fig:classifiers_performance_for_q4}). It is worth mentioning that our findings are in the same order of magnitude as findings reported in the literature. Therefore, one cannot state with confidence whether the use of the reported ML approach will bring any  benefit, as compared to a simple educated guess by the user. On the contrary, there is evidence that predictions produced under these conditions are worse than user educated guess.

\begin{figure*}[!ht]
\centering
  \includegraphics[scale=1]{figures/predictor_vs_user_error_magnitude.pdf}
  \caption{ML algorithms error magnitude (predictor versus user) for RQ3.}
    \label{fig:classifiers_performance_for_q3}
\end{figure*}

\begin{figure*}[!ht]
\centering
  \includegraphics[scale=1]{figures/predictors_performance_compared_user.pdf}
  \caption{ML algorithms performance compared to user for RQ4.}
    \label{fig:classifiers_performance_for_q4}
\end{figure*}

\subsection{Statistical Tests}\label{subsec:tests}
We can one observe in the previous tables that the performance of the three ML algorithms are very similar. To confirm whether they are really similar, we have evaluated their F-measure performances using Friedman Test\cite{Japkowicz:2011}. 
We have defined the test null hypothesis (H0) as ``the investigated algorithms have similar performance``, and to test it, we have grouped F-measure values by dataset, research question and algorithm. For each group (3 per dataset), we have calculated the p-value.    Table \ref{tab:friedman_tests_results} shows the null hypothesis(H0) can be accepted (p$-$value $>=$ 0.05) to the most cases, confirming that the investigated algorithms are similar\cite{Demsar:2006}. 

\begin{table}[!ht]
	\renewcommand{\arraystretch}{1.8}
	\caption{Friedman tests results over F-measure.}
	\label{tab:friedman_tests_results}
	\centering
	\begin{tabular}{l|c|c|c|}
		\cline{2-4}
		
		& Question & P-value & H0\\
		\cline{1-4}
% Neural Network		
%		\multicolumn{5}{ |c| }{Neural Network}\\
%		\cline{1-5} 
        \multicolumn{1}{ |c| }{\multirow{3}{*}{\rotatebox[origin=c]{90}{\scriptsize{Cassandra}}}} & Q1 & 0.135335283 & Accepted\\
		\cline{2-4}
		\multicolumn{1}{ |c| }{} & Q2 & 0.096971968 & Accepted\\
		\cline{2-4}
        \multicolumn{1}{ |c| }{} & Q3 & 0.055637998 & Accepted\\
		\hline
		\multicolumn{1}{ |c| }{\multirow{3}{*}{\rotatebox[origin=c]{90}{\scriptsize{Hadoop}}}} & Q1 & 0.223130160 & Accepted\\
		\cline{2-4}
		\multicolumn{1}{ |c| }{} & Q2 & 0.096971968 & Accepted\\
		\cline{2-4}
		\multicolumn{1}{ |c| }{} & Q3 & 0.006737947 & Reject\\
		\hline
		\multicolumn{1}{ |c| }{\multirow{3}{*}{\rotatebox[origin=c]{90}{\scriptsize{Spark}}}} 		
		& Q1 & 0.223130160 & Accepted\\
		\cline{2-4}
		\multicolumn{1}{ |c| }{} & Q2 & 0.096971968 & Accepted\\
        \cline{2-4}
		\multicolumn{1}{ |c| }{} & Q3 & 0.040762204 & Reject\\
%		\cline{1-5} 
%		\multicolumn{1}{ |c| }{} & Average & 0.7988197 & 0.7057988 & 0.7404609\\
		\hline
	\end{tabular}
\end{table}

\subsection{Discussion}\label{subsec:discussion}

Table \ref{tab:performance_summary_literature} summarizes results related to CR severity level prediction reported in the literature and those generated during our experiments. We can observe that our results are better than others \cite{Lamkanfi2010, ValdiviaGarcia2014} to address RQ1 and RQ2. we can also note that our results are better than \cite{Tian2012} and worse than \cite{Menzies2008} to address RQ3. Notice that in many cases datasets and algorithms are different. 
\begin{table}[!h]
  \centering
  \renewcommand{\arraystretch}{1.7}
  \caption{ML algorithms performance summary (cells with a pair of numbers indicate range of variation).}
    \begin{tabular}{|c|p{2.1cm}|p{1.3cm}|c|p{1.8cm}|}
\cline{2-5}    \multicolumn{1}{r|}{} & \multicolumn{1}{c|}{Research Question} & \multicolumn{1}{c|}{Project} & F-measure & \multicolumn{1}{c|}{Algorithm} \\
    \hline
    \multirow{5}[10]{*}{\begin{turn}{88}\scriptsize{Menzies\cite{Menzies2008}}\end{turn}} & \multicolumn{1}{r|}{\multirow{5}[10]{2.1cm}{Is the bug report blocker, critical, major, minor or trivial?}} & PitsA & 14.0-17.0 & Ripper \\
\cline{3-5}          &       & PitsB & 42.0-90.0 & Ripper \\
\cline{3-5}          &       & PitsC & 53.0-92.0 & Ripper \\
\cline{3-5}          &       & PitsD & 87.0-99.0 & Ripper \\
\cline{3-5}          &       & PitsE & 8.0-88.0 & Ripper \\
    \hline
    \multirow{3}[6]{*}{\begin{turn}{88} $\quad$\scriptsize{Lamkanfi\cite{Lamkanfi2010}}\end{turn}} & \multicolumn{1}{r|}{\multirow{3}[6]{2.1cm}{Is the bug report severe or non-severe?}} & Mozilla & 65.9-71.7 & Naive Bayes \\
\cline{3-5}          &       & Eclipse & 62.5-65.5 & Naive Bayes \\
\cline{3-5}          &       & GNOME & 72.7-78.5 & Naive Bayes \\
    \hline
    \multirow{6}[12]{*}{\begin{turn}{88}\scriptsize{Valdivia\cite{ValdiviaGarcia2014}}\end{turn}} & \multicolumn{1}{r|}{\multirow{6}[12]{2.1cm}{Is the bug report blocking or non-blocking?}} & Chrominum & 15.3  & Decision Tree \\
\cline{3-5}          &       & Eclipse & 15.4  & Decision Tree \\
\cline{3-5}          &       & FreeDesktop & 31.9  & Decision Tree \\
\cline{3-5}          &       & Mozilla & 42.1  & Decision Tree \\
\cline{3-5}          &       & Netbeans & 21.1  & Decision Tree \\
\cline{3-5}          &       & Netbeans & 25.6  & Decision Tree \\
    \hline
    \multirow{3}[6]{*}{\begin{turn}{88}\scriptsize{Tian\cite{Tian2012}}\end{turn}} & \multicolumn{1}{r|}{\multirow{3}[6]{2.1cm}{Is the bug report blocker, critical, major, minor or trivial?}} & OpenOffice & 12.3-74.0 & INSPect \\
\cline{3-5}          &       & Mozilla & 13.9-65.3 & INSPect \\
\cline{3-5}          &       & Eclipse & 8.6-58.6 & INSPect \\
    \hline
    \multirow{9}[18]{*}{\begin{turn}{88}\scriptsize{Ours}\end{turn}} & \multicolumn{1}{r|}{\multirow{3}[6]{2.1cm}{Will the CR severity level change?}} & Cassandra & 65.50-97.88 & Random Forest \\
\cline{3-5}          &       & Hadoop & 53.16-97.09 & Random Forest \\
\cline{3-5}          &       & Spark & 64.15-94.87 & Random Forest \\
\cline{2-5}          & \multicolumn{1}{r|}{\multirow{3}[6]{2.1cm}{Will the CR severity level increase, decrease or remain the same?\linebreak$\quad$}} & Cassandra & 42.64-97.67 & Random Forest \\
\cline{3-5}          &       & Hadoop & 18.51-97.26 & Random Forest \\
\cline{3-5}          &       & Spark & 29.90-95.35 & Random Forest \\
\cline{2-5}          & \multicolumn{1}{r|}{\multirow{3}[6]{2.1cm}{Is the bug report blocker, critical, major, minor or trivial?}} & Cassandra & 30.15-75.43 & Random Forest \\
\cline{3-5}          &       & Hadoop & 30.76-82.01 & Random Forest \\
\cline{3-5}          &       & Spark & 14.08-73.95 & Random Forest \\
    \hline
    \end{tabular}%
  \label{tab:performance_summary_literature}%
\end{table}%  





\section{Conclusions} \label{sec:conclusion}

In this paper, we have investigated the performance of  three popular ML algorithms to predict CR severity level in an imbalanced data scenario. The results based on 22901 CRs extracted from the Cassandra, Hadoop e Spark repositories have shown that Random Forest results are slightly better than the other two algorithms to predict whether the severity level will change ($RQ_1$) and whether it will increase or decrease ($RQ_2$) with good F-measure, around 0.79 and 0.62 respectively, better than findings reported in the literature.  On the other hand, results for the prediction of the final severity level on imbalanced data scenario ($RQ_3$) is similar to other results in the literature, with F-measure around 0.49. In an additional analysis conducted with Friedman Test, the three ML algorithms obtained similar performance for this experimental conditions. We have also shown that the classical measurements used in the literature do not help us deciding if the ML approach will bring any benefit to the user, and have proposed an alternative measuring approach to address this issue.

Validity threats to our research are: (a) We have assumed that user assigned severity level is correct and that there is a close relationship between it and the long description of the CR. This assumption is supported  \cite{Lamkanfi2010, Tian2012}. (b) We have considered three repositories and we have extracted 22901 CRs from it. Although we cannot generalize the results to others, the characteristics presented by Cassandra, Hadoop and Spark repositories, particularly regarding the balance of the data, are similar to those shown in the repositories studied\cite{Lamkanfi2010, Lamkanfi2011, Tian2012,ValdiviaGarcia2014}. (c) Comparison in Table \ref{tab:performance_summary_literature} is relative, since it is based on experiments run on different datasets with different algorithms. (d) Code developed in Java language and the R language for preprocessing, training, testing and analysis of results have been carefully checked may still contain bugs.

As future work, we intend to investigate other repositories and systems, and develop an approach for representing CR Systems data in a general and uniform manner, so as to facilitate the development of a general purpose ML-based Maintenance Assistant. 

%\section*{Acknowledgment}
Authors are grateful to CAPES (grant \#88881.145912/2017-01), CNPq (grant \#307560/2016-3), FAPESP (grants \#2014/12236-1, \#2015/24494-8, \#2016/50250-1, and \#2017/20945-0) and the FAPESP-Microsoft Virtual Institute (grants \#2013/50155-0, \#2013/50169-1, and \#2014/50715-9).

\input{bibliography}
\end{document}


