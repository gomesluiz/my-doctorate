%old

\begin{abstract}
In the context de Change Request (CR) systems, the severity level of a change request is considered a critical variable in the planning of the software maintenance activities, indicating as soon a CR needs to be fixed. However, the severity level assignment remains a process primarily manual depending on the experience and expertise who has reported the CR. In this paper, we present the preliminary findings of research aim to predict the severity level of a CR by analyzing its long description using text mining and a classifier based on Random Forest algorithm. The results have evidenced that this classifier can predict the severity level will change and whether severity will increase or decrease with reasonable accuracy, around 94,12\% and 93,86\% respectively. However, it has provided poor accuracy (around 67,29\%) to predict the final last severity level in imbalanced data scenario.

\end{abstract}



%new

\begin{abstract}
In the context de Change Request (CR) systems, the severity level of a change request is considered a critical variable in the planning of the software maintenance activities, indicating how soon a CR needs to be fixed. However, the severity level assignment remains primarily a manual process, mostly depending on the experience and expertise of the person who has reported the CR. In this paper, we present preliminary findings of research aimed to predict the severity level of a CR by analyzing its long description, using text mining and a classifier based on Random Forest algorithm. The results have evidenced that this classifier can predict the severity level will change and whether severity will increase or decrease with good accuracy, around 94,12\% and 93,86\% respectively. However, it has provided poor accuracy (around 67,29\%) to predict the final last severity level in imbalanced data scenario.

\end{abstract}

última frase: pensar outra abordagem. Está muito negativo. Dizer a mesma coisa de forma positiva. Comparar com a literatura?
e sobre a nova medida? e comparativo com a literatura (no abstract)?