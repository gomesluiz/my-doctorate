\documentclass[11pt,twoside]{article}

\usepackage[brazil]{babel}
\usepackage{booktabs}
\usepackage{caption}
\usepackage[dvipsnames]{xcolor}
\usepackage{enumerate}
\usepackage{float}
\usepackage{graphicx}
\usepackage[utf8]{inputenc}
\usepackage{listings}
\usepackage{pgfplots}
\usepackage{pgfplotstable}
\usepackage{rotating}
\usepackage{techrep-ic}
\usepackage{tikz}
\usepackage{textcomp}

\usetikzlibrary{shapes.geometric, arrows}

\renewcommand{\lstlistingname}{Listagem}

\pgfplotstableset{
     col sep=comma,
     columns/N/.style={column name=\#, column type={r}, string type},
     columns/Classifier/.style={column name=Classifier, column type={l}, string type},
     columns/C/.style={column name=C, column type={r}, string type},
     columns/Sigma/.style={column name=$\gamma$, column type={r}, string type},
     columns/Mtry/.style={column name=Mtry, column type={r}, string type},
     columns/Ntree/.style={int detect,column type=r,column name=Ntree},
     columns/Size/.style={int detect,column type=r,column name=Size},
     columns/Decay/.style={int detect,column type=r,column name=Decay},
     columns/Depth/.style={int detect,column type=r,column name=Depth},
     columns/Ntrees/.style={int detect,column type=r,column name=Ntrees},
     columns/Shrinkage/.style={column name=Shrink, column type={r}, string type},     
     columns/Nminobsinnode/.style={column name=Nminobs, column type={r}, string type},     
     columns/SampleMethod/.style={column name=Sample, column type={l}, string type},
     columns/Observations/.style={column name=n, column type={r}, string type},
     columns/Accuracy/.style={fixed,fixed zerofill,precision=3,column type={r}, column name=Acc},
     columns/Precision/.style={fixed,fixed zerofill,precision=3,column type={r}, column name=Precision},
     columns/Recall/.style={fixed,fixed zerofill,precision=3,column type={r}},
     columns/F1.Score/.style={fixed,fixed zerofill,precision=3,column type={r}, column name=F1Score},
     columns/Mean/.style={fixed,fixed zerofill,precision=3,column type={r}, column name=Mean},
     columns/TimeTaken/.style={column name=Time, fixed,fixed zerofill,precision=3,column type={r}},
     every head row/.style={before row=\toprule,after row=\midrule},
     every last row/.style={after row=\bottomrule},
     empty cells with={--}
}

  \tikzstyle{activity} = [rectangle, rounded corners, minimum width=10cm, minimum 
height=1cm,text centered, draw=black, fill=white]
\tikzstyle{io} = [trapezium, trapezium left angle=70, trapezium right angle=110, minimum width=3cm, minimum height=1cm, text centered, draw=black, fill=blue!30]
\tikzstyle{process} = [rectangle, minimum width=3cm, minimum height=1cm, text centered, text width=3cm, draw=black, fill=orange!30]
\tikzstyle{decision} = [diamond, minimum width=3cm, minimum height=1cm, text centered, draw=black, fill=green!30]
\tikzstyle{arrow} = [thick,->,>=stealth]
\tikzstyle{line} = [draw, -latex']

\lstset{ %
  language=R,                     % the language of the code
  basicstyle=\footnotesize,       % the size of the fonts that are used for the code
  numbers=left,                   % where to put the line-numbers
  numberstyle=\tiny\color{blue},  % the style that is used for the line-numbers
  stepnumber=1,                   % the step between two line-numbers. If it's 1, each line
                                  % will be numbered
  numbersep=5pt,                  % how far the line-numbers are from the code
  backgroundcolor=\color{white},  % choose the background color. You must add \usepackage{color}
  showspaces=false,               % show spaces adding particular underscores
  showstringspaces=false,         % underline spaces within strings
  showtabs=false,                 % show tabs within strings adding particular underscores
  %frame=no,                   % adds a frame around the code
  rulecolor=\color{black},        % if not set, the frame-color may be changed on line-breaks within not-black text (e.g. commens (green here))
  tabsize=2,                      % sets default tabsize to 2 spaces
  captionpos=t,                   % sets the caption-position to bottom
  breaklines=true,                % sets automatic line breaking
  breakatwhitespace=false,        % sets if automatic breaks should only happen at whitespace
  %Ttitle=\lstname,                 % show the filename of files included with \lstinputlisting;
                                  % also try caption instead of title
  keywordstyle=\color{black},      % keyword style
  commentstyle=\color{OliveGreen},   % comment style
  stringstyle=\color{YellowOrange},      % string literal style
  %escapeinside={\%*}{*)},         % if you want to add a comment within your code
  morekeywords={*,...},            % if you want to add more keywords to the set
  extendedchars=true,
  upquote=true,
  otherkeywords={\%}
} 

\begin{document}
\TRNumber{}
\TRYear{16}  % Dois dígitos apenas
\TRMonth{12} % Numérico, 01-12
\TRAuthor{Luiz Alberto Ferreira Gomes (RA007275)}
\TRTitle{Previsão de Mudanças na Severidade da Prioridade de um Tíquete\\ Baseada 
em Máquinas de Aprendizado}
\TRMakeCover

\pagestyle{myheadings}

\title{Previsão de Mudanças na Severidade da Prioridade de um Tíquete\\ Baseada 
em Máquinas de Aprendizado}

\author{Luiz Alberto Ferreira Gomes (RA007275) \thanks{Instituto  de Computação, Universidade
Estadual  de Campinas, 13081-970  Campinas,  SP.}}

\date{}

\maketitle

%%%%%%%%%%%%%%%%%%%%%%%%%%%%%%%%%%%%%%%%%%%%%%%%%%%%%%%%%%%%%%%%%%%%%%

\begin{abstract} 

Este relatório técnico descreve o projeto para escolha de um algoritmo de classificação para uma máquina de aprendizado que seja capaz de prever, com um níveis adequados de confiança e precisão, as modificações na severidade na prioridade
de tíquetes empregados em \textit{Bug Tracking Systems}. Com base nos dados 
fornecidos por esses sistemas,  a máquina de 
aprendizado desenvolvida terá que responder às seguintes questões: (i) se um tíquete terá a sua prioridade modificada; (ii) se essa prioridade será modificada para uma 
severidade maior ou menor e (iii) qual será a prioridade final de um tíquete ao final do seu ciclo de vida.

O resultados das experimentos realizados durante esse projeto, com base em cinco métricas de desempenho(\textit{accuracy}, \textit{precision}, 
\textit{recall}, \textit{F1 score} e \textit{execution time}), indicou que o \textit{Random Forest} é o classificador mais adequado para prever as respostas das três questões citadas acima 




%  Este trabalho é um relatório técnico de um projeto temático 
%  plurianual, visando o estudo comparativo de diversas espécies de
%  cervejas nativas do sub-continente brasileiro.  A realização deste
%  trabalho contou com o suporte financeiro do CNPq e FAPESP, e foi
%  imensamente facilitada pela infraestrutura de pesquisa cervisíaca
%  instalada no Campus da UNICAMP.
%q
%  Com base nessas pesquisas, determinamos que a altura da cerveja $h$q
%  e a altura da espuma $e$ satifazem aproximadamente a inequação
%  $(\sqrt{e^2 + h^2 + 2 h e})^3 \leq \exp(3 \log K_0^\ast)$, onde
%  $K_0^\ast$ é a altura do copo.  Esta fórmula é válida,
%  aparentemente, inclusive para espécies mais pigmentadas, como {\em
%  Malzbier}.  Em vista disso, e dos resultados análogos obtidos por
%  A. B. Stémio em experiências com {\em Guaraná} e $x$-{\em Cola},
%  conjeturamos que a fórmula pode ser aplicada (com pequenas
%  modificações) também a {\em Champagne} e outros líquidos de
%  composição similar.
\end{abstract}

\section{Introdução}
	Tíquetes(em inglês:\textit{issue}) são utilizados em \textit{Bug Tracking Systems}(BTS) para registrar requisições de modificação para um 
	determinado produto de software. Tais pedidos podem contemplar a inclusão ou modificação de requisitos, a correção de \textit{bugs} encontrados 
	e ou adaptações para uma nova plataforma operacional\cite{SOM}. Sendo que cada um, no momento da sua criação, recebe uma prioridade que varia 
	em escala de severidade(\textit{blocker}, \textit{critical}, \textit{major}, \textit{minor} ou \textit{trivial}) que auxilia a equipe de manutenção 
	no planejamento das suas atividades.
	
	No decorrer do ciclo de vida do tíquete (Figura \ref{fig:ciclo-de-vida}), a sua prioridade pode mudar $-$ tanto para uma severidade maior quanto para uma menor. Tais mudanças afetam o planejamento da manutenção de um produto, fazendo com que a equipe responsável tenha que refazê-lo inúmeras vezes.

	\begin{figure}[hbt!]  
  \centering
  \begin{tikzpicture}[node distance=1.5cm]
    \node (a1) [activity] {Open};
    \node (a2) [activity, below of=a1] {In Progress};
    \node (a3) [activity, below of=a2] {Resolved};
    \node (a4) [activity, below of=a3] {Closed};
    \node (a5) [activity, below of=a4] {Reopened};
    
    \draw [arrow] (a1) -- (a2);
    \draw [arrow] (a2) -- (a3);
    \draw [arrow] (a3) -- (a4);
    \draw [arrow] (a4) -- (a5);
    \draw [arrow] (a5.east)  -- ++(1.5,0) |- (a1.east);
    \draw [arrow] (a3.west)  -- ++(-1.5,0) |- (a5.west);
  \end{tikzpicture}
  \caption{Ciclo de vida de um tíquete\cite{SOM}.}
  \label{fig:ciclo-de-vida}
\end{figure}

%  As propriedades lúdicas, mnemolíticas e catalógicas de 
%  soluções fraca- e medianamente concentradas de 1-metil-metanol
%  em hidróxido de hidrônio aquoso ($\rm H_3O^{+} HO^{-}\cdot {\mathit n}H_2O$)
%  tem sido objeto de intensos estudos experimentais por
%  cientistas no mundo todo~\cite{AHU,KNU}.
%  
%  Durante os últimos 20 anos, os autores coordenaram uma equipe
%  multidiplinar de pesquisadores, na UNICAMP e em outras instituições,
%  cujo objetivo derradeiro é elucidar e quantificar a relação entre...
O contéudo relatado neste documento é parte de um trabalho de pesquisa de doutorado mais abrangente cujo objetivo é desenvolver um assistente 
para manutenção de software livre baseado em máquinas de aprendizado.

\section{Objetivos do projeto}
Considerando o contexto acima, o objetivo deste projeto é selecionar, com base em métricas conhecidas\cite{JON}, classificadores para a construção de uma máquina de aprendizado $-$ baseada 
em um modelo de dados extraído de repositórios de BTS $-$ que responda adequadamente às seguintes questões:
	\begin{enumerate}[Q1.]
  		\item A prioridade de um dado tíquete mudará durante o seu ciclo de vida?
  		\item A prioridade de um dado tíquete mudará para uma severidade mais alta ou mais baixa durante o seu ciclo de vida?
  		\item Qual será a prioridade de um dado tíquete ao final do seu ciclo de vida?
	\end{enumerate}
	
\pagebreak
\section{Metodologia utilizada}
O projeto descrito neste relatório técnico foi baseou-se nas atividades descritas na Figura \ref{fig:metodologia}.

\begin{figure}[!ht]	
	\centering
	\begin{tikzpicture}[node distance=1.5cm]
		\node (a1) [activity] {Elaboração do modelo de dados};
		\node (a2) [activity, below of=a1] {Extração e pré-processamento de dados};
		\node (a3) [activity, below of=a2] {Treinamento e testes da máquina de aprendizado};
		\node (a4) [activity, below of=a3] {Análise dos Resultados};
		
		\draw [arrow] (a1) -- (a2);
		\draw [arrow] (a2) -- (a3);
		\draw [arrow] (a3) -- (a4);
		\draw [arrow] (a4.east)  -- ++(2.5,0) |- (a1.east);
	\end{tikzpicture}
	\caption{Metodologia utilizada no projeto.}
	\label{fig:metodologia}
\end{figure}

\section{Elaboração do modelo de dados}
O processo de elaboração modelo de dados preliminar (apresentado na Tabela 
\ref{tab:modelo-preliminar}) para satisfazer os requisitos do projeto 
baseou-se nas seguintes atividades:
\begin{enumerate}
  \item \textbf{Análise dos dados dos tíquetes:} essa atividade baseou-se nos 
  campos estruturados da base de dados do BTS\cite{ATL}. A partir dessa 
análise os campos de 1 a 7 do modelo de dados foram definidos. 
  \item \textbf{Análise dos termos do campo descrição}: essa atividade 
baseou-se na mineração de termos (Figura \ref{fig:termos}) relacionados à manutenção de software do texto 
da descrição do tíquete\cite{ANT}. A biblioteca \texttt{TextMinning}\cite{WIL} 
da linguagem R foi utilizada para a extração e visualização desses termos. A 
partir dessa análise os campos de 8 a 21 do modelo de dados foram definidos. 

  \item \textbf{Análise do relacionamento entre tíquetes}: os relacionamentos 
entre os tíquetes foram definidos a partir da análise dos \textit{hiperlinks} 
$-$ para outros tíquetes $-$ escritos no texto da descrição de um tíquete. A 
partir dessa análise os campos de 23 a 27 do modelo de dados foram definidos.
  \item \textbf{Análise do histórico de mudanças dos tíquetes}: as mudanças na 
  prioridade de cada tíquete foram analisadas com base no histórico de 
mudanças de cada um. Essa análise derivou os campos de 28 e 29 do modelo de 
dados.
\end{enumerate}

\begin{figure}[!ht]
	\centering
  \includegraphics[scale=0.4]{/Users/luiz/Workspace/issue-miner/data/png/plot-termos}
  \caption{Frequência dos termos de manutenção das descrições dos tíquetes.}
  \label{fig:termos}
\end{figure}


\begin{table}[ht]\centering
  \tiny
  \renewcommand{\arraystretch}{1.2}
  \begin{tabular}{@{}llp{9cm}p{2cm}l@{}}\toprule
    \#  &  Nome  &  Descrição  &  Tipo  &  Escala \\
    \midrule 
    1  &  IssueKey  &  Identificador do tíquete  &  Qualitativo  &  Nominal \\
    2  &  Type  &  Tipo do tíquete: \textit{Bug} ,  \textit{New Feature} ,  \textit{Improvement} ,  \textit{Task} ,  \textit{Sub-task} ou \textit{Wish}  &  Qualitativo  &  Nominal \\
    3  &  Priority  &  Prioridade(ou severidade) do tíquete: \textit{Blocker} ,  \textit{Critical} ,  \textit{Major} ,  \textit{Minor} ou \textit{Trivial}  &  Qualitativo  &  Nominal \\
    4  &  DayToResolve  &  Quantidade de dias que foram utilizados para resolvê-lo  &  Quantitativo discreto  &  Racional \\
    5  &  Resolution  &  Tipo de resolução dada: \textit{Fixed} ,  \textit{Duplicate} ,  \textit{Unresolved} ,  \textit{Won’t Fix} ,  \textit{Cannot Reproduce}, \textit{Not a Problem}, 
    \textit{Later}, \textit{Invalid} ou \textit{Implemented} &  Qualitativo  &  Nominal\\
	6 & Status & Situação do tíquete: \textit{Close} ou  \textit{Resolved}. & Qualitativo & Nominal\\
	7 & QuantityOfComments & Quantidade de comentários feito sobre um tíquete. & Quantitativo discreto & Racional\\
	8 & HasBlockWord & Indica se a palavra “\textit{block}” está na descrição do tíquete. & Qualitativo & Nominal\\
	9 & HasChangeWord & Indica se a palavra “\textit{change}” está na descrição do tíquete. & Qualitativo & Nominal\\
	10 & HasDebugWord & Indica se a palavra “\textit{debug}” está na descrição do tíquete. & Qualitativo & Nominal\\
	11 & HasErrorWord & Indica se a palavra “\textit{error}” está na descrição do tíquete. & Qualitativo & Nominal\\
	12 & HasExceptionWord & Indica se a palavra “\textit{exception}” está na descrição do tíquete. & Qualitativo & Nominal\\
	13 & HasFailWord & Indica se a palavra “\textit{fail}” está na descrição do tíquete. & Qualitativo & Nominal\\
	14 & HasFixWord & Indica se a palavra “\textit{fix}” está na descrição do tíquete. & Qualitativo & Nominal\\
	15 & HasIssueWord & Indica se a palavra “\textit{issue}” está na descrição do tíquete. & Qualitativo & Nominal\\
	16 & HasPatchWord & Indica se a palavra “\textit{patch}” está na descrição do tíquete. & Qualitativo & Nominal\\
	17 & HasProblemWord & Indica se a palavra “\textit{problem}” está na descrição do tíquete. & Qualitativo & Nominal\\
	18 & HasSupportWord & Indica se a palavra “\textit{support}” está na descrição do tíquete. & Qualitativo & Nominal\\
	19 & HasTestWord & Indica se a palavra “\textit{test}” está na descrição do tíquete. & Qualitativo & Nominal\\
	20 & HasThrowsWord & Indica se a palavra “\textit{throws}” está na descrição do tíquete. & Qualitativo & Nominal\\
	21 & HasWarnWord & Indica se a palavra “\textit{warn}” está na descrição do tíquete. & Qualitativo & Nominal\\
	23 & DepthOfTree & Profundidade da árvore onde se encontra o tíquete. & Quantitativo discreto & Racional\\
	24 & QuantityOfChildren & Quantidade de filhos que o tíquete possui. & Quantitativo discreto & Racional\\
	25 & WeightOfChildren & Peso total ponderado dos filhos que o tíquete possui. & Quantitativo contínuo & Racional\\
	26 & QuantityOfParents & Quantidade de parentes que o tíquete possui. & Quantitativo discreto & Racional\\
	27 & WeightOfParents & Peso total ponderado dos parentes que o tíquete possui. & Quantitativo contínuo & Racional\\
	28 & ChangedInPriority & Indica se houve mudança na prioridade do tíquete(1) ou não houve(0)& Qualitativo & Nominal\\    
	29 & ChangedInPriorityTo & Indica se houve mudança na prioridade do tíquete para baixo(1), para cima (2) ou não houve mudança (0)& Qualitativo & Nominal\\
    \bottomrule 
  \end{tabular}
  \caption{Modelo de dados preliminar.}
  \label{tab:modelo-preliminar}
\end{table}

\section{Extração e pré-processamento dos dados}
A execução desta etapa produziu um arquivo no formato CSV $-$ tecnicamente correto e consistente\cite{JON} $-$ contendo o modelo de dados elaborado na etapa anterior, e foi codificada em dois módulos com responsabilidades distintas: o (i) de extração de dados e o (ii) de pré-processamento.

O módulo de extração dos dados é constituído pelo programa, escrito na linguagem Java, \texttt{IssueBaseDataExtractor}\footnote{O código fonte do \texttt{IssueBaseDataExtractor} não foi apresentado neste documento por não fazer parte do escopo da disciplina e por ser um código razoavelmente extenso.} e pelo \textit{script}, escrito na linguagem R, \texttt{IssueHistoryData Extractor} (Listagem \ref{lst:IssueHistoryDataExtractor}). O primeiro tem a responsabilidade de extrair os dados básicos do tíquete a partir do repositório remoto do BTS Jira do software HADOOP $-$ utilizado como repositório de testes deste projeto $-$ que disponibiliza os dados no formato XML. O segundo tem a responsabilidade de capturar os histórico de mudanças de um tíquete e extrair a partir de uma tabela contendo as mudanças de prioridades ocorridas. Diferentemente dos dados básicos dos tíquetes, os dados do histórico de mudanças são disponibilizados pelo BTS Jira no formato HTML.

O módulo de pré-processamento é constituído pelo \textit{script}, escrito na linguagem R, \texttt{IssueDataPreProcessor} (Listagem \ref{lst:IssueDataPreProcessor}) cujas responsabilidades são as seguintes: (i) mesclar os dados básicos dos tíquetes com o histórico de mudanças de prioridade; (ii) remover as colunas irrelevantes (como a coluna \texttt{IssueKey}, utilizada para a mesclagem do histórico); e (iii) converter dados categóricos em numéricos(utilizando os métodos de conversão categórico ordinal para numérico ordinal (campo \texttt{Priority}) e o método \textit{one-hot} (campos \texttt{Type}, \texttt{Resolution} e \texttt{Status})\cite{FAC}. O pacote \texttt{Dummies} da linguagem R foi utilizado para implementar o método \textit{one-hot}\cite{DUM}.


A Tabela \ref{tab:modelo-final} apresenta o modelo final gerado pelos \textit{scripts} de  pré-processamento sobre os arquivos de dados extraídos da base de dados do BTS HADOOP.

\begin{table}[ht]\centering
  \tiny
  \renewcommand{\arraystretch}{1.2}
  \begin{tabular}{@{}llp{9cm}p{2cm}l@{}}\toprule
    \#  &  Nome  &  Descrição  &  Tipo  &  Escala \\
    \midrule 
    1  &  Type\_Bug  &  Indica um \textit{Bug}(1) ou não(0). &  Qualitativo  &  Nominal \\
    2  &  Type\_Improvement  &  Indica um \textit{Improvement}(1) ou não(0). &  Qualitativo  &  Nominal \\
    3  &  Type\_NewFeature  &  Indica um \textit{NewFeature}(1) ou não(0). &  Qualitativo  &  Nominal \\
    4  &  Type\_Subtask  &  Indica um \textit{Subtask}(1) ou não(0). &  Qualitativo  &  Nominal \\
    5  &  Type\_Task  &  Indica um \textit{Task}(1) ou não(0). &  Qualitativo  &  Nominal \\
    6  &  Type\_Test  &  Indica um \textit{Test}(1) ou não(0). &  Qualitativo  &  Nominal \\
	7  &  Type\_Wish  &  Indica um \textit{Wish}(1) ou não(0). &  Qualitativo  &  Nominal \\
    8  &  DayToResolve  &  Quantidade de dias que foram utilizados para resolvê-lo  &  Quantitativo discreto  &  Racional \\
    9  &  Resolution\_CannotReproduce  &  Indica um tipo de resolução \textit{CannotReproduce}(1) ou não(0)  &  Qualitativo  &  Nominal\\
    10  &  Resolution\_Duplicate  &  Indica um tipo de resolução \textit{Duplicate}(1) ou não(0)  &  Qualitativo  &  Nominal\\
    11  &  Resolution\_Fixed  &  Indica um tipo de resolução \textit{Fixed}(1) ou não(0)  &  Qualitativo  &  Nominal\\
    12  &  Resolution\_Implemented  &  Indica um tipo de resolução \textit{Implemented}(1) ou não(0)  &  Qualitativo  &  Nominal\\
    13  &  Resolution\_Incomplete  &  Indica um tipo de resolução \textit{Incomplete}(1) ou não(0)  &  Qualitativo  &  Nominal\\
    14  &  Resolution\_Invalid  &  Indica um tipo de resolução \textit{Invalid}(1) ou não(0)  &  Qualitativo  &  Nominal\\
 	15  &  Resolution\_Later  &  Indica um tipo de resolução \textit{Later}(1) ou não(0)  &  Qualitativo  &  Nominal\\
    16  &  Resolution\_NotAProblem  &  Indica um tipo de resolução \textit{Not A Problem}(1) ou não(0)  &  Qualitativo  &  Nominal\\
    17  &  Resolution\_Unresolved  &  Indica um tipo de resolução \textit{Unresolved}(1) ou não(0)  &  Qualitativo  &  Nominal\\
    18  &  Resolution\_WontFix  &  Indica um tipo de resolução \textit{Wont Fix}(1) ou não(0)  &  Qualitativo  &  Nominal\\
    
	19 & Status\_Closed & Indica um \textit{Status} \textit{Closed} (1) ou não. & Qualitativo & Nominal\\
	20 & Status\_Resolved & Indica um \textit{Status} \textit{Resolved} (1) ou não. & Qualitativo & Nominal\\
	
	
	21 & QuantityOfComments & Quantidade de comentários feito sobre um tíquete. & Quantitativo discreto & Racional\\
	22 & HasBlockWord & Indica se a palavra “\textit{block}” está na descrição do tíquete. & Qualitativo & Nominal\\
	23 & HasChangeWord & Indica se a palavra “\textit{change}” está na descrição do tíquete. & Qualitativo & Nominal\\
	24 & HasDebugWord & Indica se a palavra “\textit{debug}” está na descrição do tíquete. & Qualitativo & Nominal\\
	25 & HasErrorWord & Indica se a palavra “\textit{error}” está na descrição do tíquete. & Qualitativo & Nominal\\
	26 & HasExceptionWord & Indica se a palavra “\textit{exception}” está na descrição do tíquete. & Qualitativo & Nominal\\
	27 & HasFailWord & Indica se a palavra “\textit{fail}” está na descrição do tíquete. & Qualitativo & Nominal\\
	28 & HasFixWord & Indica se a palavra “\textit{fix}” está na descrição do tíquete. & Qualitativo & Nominal\\
	29 & HasIssueWord & Indica se a palavra “\textit{issue}” está na descrição do tíquete. & Qualitativo & Nominal\\
	30 & HasPatchWord & Indica se a palavra “\textit{patch}” está na descrição do tíquete. & Qualitativo & Nominal\\
	31 & HasProblemWord & Indica se a palavra “\textit{problem}” está na descrição do tíquete. & Qualitativo & Nominal\\
	32 & HasSupportWord & Indica se a palavra “\textit{support}” está na descrição do tíquete. & Qualitativo & Nominal\\
	33 & HasTestWord & Indica se a palavra “\textit{test}” está na descrição do tíquete. & Qualitativo & Nominal\\
	34 & HasThrowsWord & Indica se a palavra “\textit{throws}” está na descrição do tíquete. & Qualitativo & Nominal\\
	35 & HasWarnWord & Indica se a palavra “\textit{warn}” está na descrição do tíquete. & Qualitativo & Nominal\\
	36 & DepthOfTree & Profundidade da árvore onde se encontra o tíquete. & Quantitativo discreto & Racional\\
	37 & QuantityOfChildren & Quantidade de filhos que o tíquete possui. & Quantitativo discreto & Racional\\
	38 & WeightOfChildren & Peso total ponderado dos filhos que o tíquete possui. & Quantitativo contínuo & Racional\\
	39 & ChangedInPriority & Indica se houve mudança na prioridade do tíquete(1) ou não houve(0)& Qualitativo & Nominal\\    
	40 & ChangedInPriorityTo & Indica se houve mudança na prioridade do tíquete para baixo(1), para cima (2) ou não houve mudança (0)& Qualitativo & Nominal\\
	41  &  PriorityCode  &  Prioridade(ou severidade) do tíquete: \textit{Blocker}(5) ,  \textit{Critical}(4) ,  \textit{Major}(3) ,  \textit{Minor}(2) ou \textit{Trivial}(1)  &  Qualitativo  &  Nominal \\
 
    \bottomrule 
  \end{tabular}
  \caption{Modelo de dados final.}
  \label{tab:modelo-final}
\end{table}


\section{Treinamento e testes da máquina de aprendizado}
A execução desta etapa produziu três arquivos no formato CSV contendo as métricas de desempenho, bem como os valores dos hiperparâmetros $-$ escolhidos apos o \textit{cross validation} 
com 3-\textit{fold} $-$ utilizados pelos classificadores que foram implementados pelos \textit{scripts} \texttt{Predict-Q1.R} (Listagem \ref{lst:Predict-Q1}), \texttt{Predict-Q2.R} (Listagem \ref{lst:Predict-Q2}) e \texttt{Predict-Q3.R} (Listagem \ref{lst:Predict-Q3}). O conteúdo 
desses arquivos, apresentados nas Tabelas \ref{tab:desempenho-para-q1}, \ref{tab:desempenho-para-q2} e \ref{tab:desempenho-para-q3}, será analisado na próxima etapa com o objetivo de 
selecionar o classificador mais adequado para responder, respectivamente, a cada  uma das questões de 
pesquisa propostas neste documento. 

A partir do arquivo gerado na etapa de pré-processamento, contendo o modelo e os dados propriamente ditos, cada um dos \textit{scripts} executou os algoritmos \textit{Gradiente Boosting Machine}(GBM), 
\textit{Neural Networks}(NNet), \textit{Random Forest}(RF) e \textit{Support Vector Machine}(SVM) utlizando o pacote \texttt{Caret}\cite{CAR} da linguagem R. A Tabela \ref{tab:hiperparametros} apresenta 
os hiperparâmetros avaliados durante a execução de cada um cujos valores foram os mesmos sugeridos nas atividades desta disciplina.

\begin{table}[!ht]\centering
	\begin{tabular}{@{} p{3cm}  p{12cm}@{}}
		\toprule
		\textbf{Classificador} & \textbf{Hiperparâmetros} \\ 
		\midrule
    GBM & Número de árvores = 30, 70, e 100, Learning rate = 0.1 e 0.05 e 
Profundidade das arvores = 5.\\ 
		Nnet & Neurônios na camada escondida= 10, 20, 30 e 40.\\ 
    RF & mtry = 10, 15, 20, 25 e ntrees = 100, 200, 300 e 400.\\
		SVM & C=$2^{-5}$, $2^{0}$, $2^{5}$, $2^{10}$ e sigma= $2^{-15}$, $2^{-10}$, 
$2^{-5}$, $2^{0}$, $2^5$.\\
		\bottomrule
	\end{tabular}
	 \caption{Hiperpâmetros candidatos para classificadores.}
  	 \label{tab:hiperparametros}
\end{table}

Por sua vez, cada classificador recebeu amostras de 1000 a 6000 tíquetes (em intervalos de 1000 elementos) em três ciclos de execução $-$ um para cada método de amostragem 
empregado: (i) randômico (denominado de "\textit{random}") cujos elementos de cada classe da amostra são escolhidas aleatoriamente; (ii) partes iguais (denominado de "1/2")
cujos elementos de cada classe da amostra são distribuídos em partes iguais; e o (iii) proporcional (denominado de "proportional") cujos elementos da amostra são distribuídos 
proporcionalmente com relação ao conjunto de dados originais\cite{JAS1}. 

O treinamento e os testes dos classificadores foram relacionados em um computador com o processador \textit{Intel Core} I5 (com dois núcleos)  e com 8G de memória RAM.

\begin{sidewaystable}[ph!]
  \begin{center}
  \tiny
  \pgfplotstableread{/Users/luiz/Workspace/issue-miner/data/csv/predict-q1-result.csv}{\loadedtable}
  \pgfplotstabletypeset[]{\loadedtable}
  \end{center}
  \caption{Desempenho dos classificadores para a questão Q1.}
  \label{tab:desempenho-para-q1}
\end{sidewaystable}

\begin{sidewaystable}[ph!]
  
  \begin{center}
  \tiny
  \pgfplotstableread{/Users/luiz/Workspace/issue-miner/data/csv/predict-q2-result.csv}{\loadedtable}
  \pgfplotstabletypeset[]{\loadedtable}
  \end{center}
  \caption{Desempenho dos classificadores para a questão Q2.}
  \label{tab:desempenho-para-q2}
\end{sidewaystable}

\begin{sidewaystable}[ph!]
  \begin{center}
  \pgfplotstableread{/Users/luiz/Workspace/issue-miner/data/csv/predict-q3-result.csv}{\loadedtable}
  \tiny
  \pgfplotstabletypeset[]{\loadedtable}
  \end{center}
  \caption{Desempenho dos classificadores para a questão Q3.}
  \label{tab:desempenho-para-q3}
\end{sidewaystable}

\pagebreak
\section{Análise dos resultados}
A execução desta etapa resultou na escolha do classificador, do tamanho 
da amostra e do método de amostragem mais apropriado para responder à 
cada questões de pesquisa proposta neste projeto. Essa análise baseou-se nas 
métricas de acurácia, precisão, \textit{recall} e F1 \textit{score}\cite{JAS} e
no tempo de execução gasto por cada classificador para o aprendizado e teste de
cada amostra. Os gráficos com os valores  dessas métricas, apresentados abaixo, foram gerados pelo \textit{script} \texttt{IssueDataGraphics.R} (Listagem \ref{lst:IssueDataGraphics}).

\subsection{Desempenhos dos classificadores na questão Q1.}
\begin{figure}[!ht]
  \centering
  \includegraphics[scale=0.5]{/Users/luiz/Workspace/issue-miner/data/png/plot-q1-accuracy.png}
  \caption{Acurácia dos classificadores para a questão Q1.}
\end{figure}

\pagebreak
\begin{figure}[!ht]
  \centering
  \includegraphics[scale=0.5]{/Users/luiz/Workspace/issue-miner/data/png/plot-q1-precision}
  \caption{Precisão dos classificadores para a questão Q1.}
\end{figure}

\pagebreak
\begin{figure}[!ht]
  \centering
  \includegraphics[scale=0.5]{/Users/luiz/Workspace/issue-miner/data/png/plot-q1-recall}
  \caption{Recall dos classificadores para a questão Q1.}
\end{figure}

\pagebreak
\begin{figure}[!ht]
  \includegraphics[scale=0.5]{/Users/luiz/Workspace/issue-miner/data/png/plot-q1-F1Score}
  \caption{F1 Score dos classificadores para a questão Q1.}
\end{figure}

\pagebreak
\begin{figure}[!ht]
  \centering
  \includegraphics[scale=0.5]{/Users/luiz/Workspace/issue-miner/data/png/plot-q1-TimeTaken}
  \caption{Tempo de execução dos classificadores para a questão Q1.}
\end{figure}

\pagebreak
\subsection{Desempenhos dos classificadores na questão Q2}
\begin{figure}[!ht]
\centering
  \includegraphics[scale=0.5]{/Users/luiz/Workspace/issue-miner/data/png/plot-q2-accuracy.png}
  \caption{Acurácia dos classificadores para a questão Q2.}
\end{figure}

\pagebreak
\begin{figure}[!ht]
\centering
  \includegraphics[scale=0.5]{/Users/luiz/Workspace/issue-miner/data/png/plot-q2-precision}
  \caption{Precisão dos classificadores para a questão Q2.}
\end{figure}

\pagebreak
\begin{figure}[!ht]
\centering
  \includegraphics[scale=0.5]{/Users/luiz/Workspace/issue-miner/data/png/plot-q2-recall}
  \caption{Recall dos classificadores para a questão Q2.}
\end{figure}

\pagebreak
\begin{figure}[!ht]
\centering
  \includegraphics[scale=0.5]{/Users/luiz/Workspace/issue-miner/data/png/plot-q2-F1Score}
  \caption{F1 Score dos classificadores para a questão Q2.}
\end{figure}

\pagebreak
\begin{figure}[!ht]
\centering
  \includegraphics[scale=0.5]{/Users/luiz/Workspace/issue-miner/data/png/plot-q2-TimeTaken}
  \caption{Tempo de execução dos classificadores para a questão Q2.}
\end{figure}

\pagebreak
\subsection{Desempenhos dos classificadores na questão 3}
\begin{figure}[!ht]
\centering
  \includegraphics[scale=0.5]{/Users/luiz/Workspace/issue-miner/data/png/plot-q3-accuracy.png}
  \caption{Acurácia dos classificadores para a questão Q3.}
\end{figure}

\pagebreak
\begin{figure}[!ht]
\centering
  \includegraphics[scale=0.5]{/Users/luiz/Workspace/issue-miner/data/png/plot-q3-precision}
  \caption{Precisão dos classificadores para a questão Q3.}
\end{figure}

\pagebreak
\begin{figure}[!ht]
\centering
  \includegraphics[scale=0.5]{/Users/luiz/Workspace/issue-miner/data/png/plot-q3-recall}
  \caption{Recall dos classificadores para a questão Q3.}
\end{figure}

\pagebreak
\begin{figure}[!ht]
\centering
  \includegraphics[scale=0.5]{/Users/luiz/Workspace/issue-miner/data/png/plot-q3-F1Score}
  \caption{F1 Score dos classificadores para a questão Q3.}
\end{figure}

\pagebreak
\begin{figure}[!ht]
\centering
  \includegraphics[scale=0.5]{/Users/luiz/Workspace/issue-miner/data/png/plot-q3-TimeTaken}
  \caption{Tempo de execução dos classificadores para a questão Q3.}
\end{figure}


\pagebreak
A avaliação as métricas produzidas pelos classificadores de cada máquina de 
aprendizado escritas para responder as questões propostas resultou em:

\begin{enumerate}
  \item O classificador \textit{Random Forest} (com Mtry igual a 15 e Ntrees igual a 200) mostrou-se mais adequado para tratar a questão Q1. Esse classificador com 3000 tíquetes, amostrados pelo método "1/2", alcançou o menor tempo de execução , em torno de 1.15 segundos (Tabela \ref{tab:desempenho-para-q1}). A média obtida de {0.918} das métricas analisadas, embora não seja a maior, pode ser considerada também razoável para o problema em questão.
  \item O classificador \textit{Random Forest} (com Mtry igual a 15 e Ntrees igual a 200) mostrou-se também mais adequado para tratar a questão Q2. Esse classificador com 3000 tíquetes, amostrados pelo método "1/2", alcançou o menor tempo de execução , em torno de 1.025 segundos (Tabela \ref{tab:desempenho-para-q2}). A média obtida de {0.912} das métricas analisadas, embora não seja a maior, pode ser considerada também razoável para o problema em questão. O \textit{Gradiente Boosting Machine} não poder ser executado para a questão Q2 em razão de erros ocorridos cujas soluções não foram encontradas pelo autor deste relatório.
  \item O classificador \textit{Random Forest} (com Mtry igual a 25 e Ntrees igual a 200), assim como nas questões Q1 e Q2, mostrou-se mais adequado para tratar a questão Q3. Esse classificador com 6000 tíquetes, amostrados pelo método "random", embora não alcançado o menor tempo de execução , em torno de 2.744 segundos (Tabela \ref{tab:desempenho-para-q3}), obteve a melhor média {0.727} das métricas analisadas, ainda sim, uma média considerada baixa para o problema em questão; o que exigirá mais refinamentos no modelo de dados. O \textit{Gradiente Boosting Machine}, bem como os métodos de amostragem "1/2" e "proportional, não puderam ser executados para a questão Q3 em razão de erros ocorridos cujas soluções não foram encontradas pelo autor deste relatório.
\end{enumerate}

\section{Conclusão}
Este relatório apresentou o processo para a escolha de um algoritmo para a 
implementação de uma maquina de aprendizado capaz de prever alterações na 
severidade da prioridade de um tíquete empregados em sistemas de BTS. Esse processo de escolha
envolveu a elaboração do modelo de dados para responder as três questões, a extração dos dados
da base do BTS, o pre-processamento desses dados, o treinamento 
e testes de quatro algoritmos de classificação e a análise do desempenho 
desses classificadores. 

A escolha levou em consideração, de forma equilibrada, não somente a acurácia dos classificadores, 
mas também outras três métricas referenciadas na literatura (a precisão, o \textit{recall} e o \textit{F1.Score}), além do tempo de execução dos classificadores executando em computador padrão atual de mercado. Neste sentido buscou-se uma média entre as métricas em 
torno de 0.9 com o menor tempo de resposta o menor sempre que possível possível. O \textit{Random Forest} foi o classificador que melhor atendeu a esses requisitos. 

Alguns problemas alguns relacionados aos erros apresentados com classificador
como o \textit{Gradiente Boosting Machine} (na questão Q2 e Q3) e com os métodos de amostragens "1/2" e "proportional" (na questão Q3) não foram solucionados ate o termino deste relatório, mesmo após diversas tentativas.


\begin{thebibliography}{99}
\bibitem{ANT} ANTONIOL, G., et. al.  {\bf Is It a Bug or an Enhancement?: A Text-based Approach to Classify Change Requests.}. Proceedings of the 2008 Conference of the 
Center for Advanced Studies on Collaborative Research: Meeting of Minds. Ontario(CDN):ACM, 2008.
\bibitem{ATL} ATLASSIAN.  {\bf JIRA User's Guide}. Disponível em: https://confluence.atlassian.com/jira064/jira-user-s-guide-720416011.html.
Acessado em 10 de novembro de 2016.
\bibitem{CAR} CARET.  {\bf The Caret Package}. Disponível em: 
http://topepo.github.io/caret/index.html. Acessado em 10 de novembro de 2016.
\bibitem{DUM} CRAN.  {\bf The Dummies Package}. Disponível em: 
https://cran.r-project.org/web/packages/dummies/dummies.pdf. Acessado em 10 de novembro de 2016.

\bibitem{FAC} FACELLI, K. et.al. {\bf Inteligência Artificial: Uma Abordagem de Aprendizado de Máquina}. Rio de Janeiro: LTC, 2015.
\bibitem{JAS} BROWNLEE, J.  {\bf Classification Accuracy is Not Enough: More 
Performance Measures You Can Use}. Disponível em: 
http://machinelearningmastery.com/classification-accuracy-is-not-enough-more-per
formance-measures-you-can-use/.Acessado em 10 de novembro de 2016.
\bibitem{JAS1} BROWNLEE, J.  {\bf 8 Tactics to Combat Imbalanced Classes in Your Machine Learning Dataset - Machine Learning Mastery}. Disponível em: 
http://machinelearningmastery.com/tactics-to-combat-imbalanced-classes-in-your-machine-learning-dataset/. Acessado em 10 de novembro de 2016.

\bibitem{JON} JONGE, E., LOO, M. {\bf An Introduction do Data Cleaning with R}. Hague(NL): Statistics Netherlands, 2013.
\bibitem{SOM} SOMMERVILLE, I. {\bf Engenharia de Software.} 9.ed. São Paulo: Pearson Prentice Hall, 2011.
\bibitem{WIL} WILLIAMS, G. {\bf Hands-On Data Sciense with R Text Mining}. Disponível em: http://HandsOnDataScience.com/.
Acessado em 15 de novembro de 2016. 

\end{thebibliography}
\pagebreak

\appendix
\section{Apêndices}
\subsection{Códigos fontes em R}

\lstinputlisting[language=R, caption=IssueHistoryDataExtractor.R, label=lst:IssueHistoryDataExtractor]{/Users/luiz/Workspace/issue-miner/scripts/extract_issues_history.R}

%\lstinputlisting[language=R, caption=IssueHistoryDataExtractor.R, label=lst:IssueHistoryChangesExtractor]{/Users/luiz/Workspace/issue-miner/scripts/extract_history_changes.R}

\lstinputlisting[language=R, caption=IssueDataPreProcessor.R, label=lst:IssueDataPreProcessor]{/Users/luiz/Workspace/issue-miner/scripts/preprocess_issues_data.R}

\lstinputlisting[language=R, caption=Predict-Q1.R, label=lst:Predict-Q1]{/Users/luiz/Workspace/issue-miner/scripts/predict-q1.R}

\lstinputlisting[language=R, caption=Predict-Q2.R, label=lst:Predict-Q2]{/Users/luiz/Workspace/issue-miner/scripts/predict-q2.R}

\lstinputlisting[language=R, caption=Predict-Q3.R, label=lst:Predict-Q3]{/Users/luiz/Workspace/issue-miner/scripts/predict-q3.R}

\lstinputlisting[language=R, caption=input\_output\_data.R, label=lst:inputoutputdata]{/Users/luiz/Workspace/issue-miner/scripts/lib/input_output_data.R}

\lstinputlisting[language=R, caption=train\_predict\_data.R, label=lst:trainpredictdata]{/Users/luiz/Workspace/issue-miner/scripts/lib/train_predict_data.R}

\lstinputlisting[language=R
	, caption=IssueDataGraphics.R
	, label=lst:IssueDataGraphics]{/Users/luiz/Workspace/issue-miner/scripts/issues_data_graphics.R}


\end{document}
