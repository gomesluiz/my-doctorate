\section{Desafios da Área de Estudo}
\begin{frame}[t, fragile]{Desafios Gerais}
    \begin{itemize}
      \item O estudo da manutenção de software é bastante \alert{desafiador} pois lida com fatores  técnicos e humanos.
%      \begin{itemize}
%	   \item técnicos: modularidade, acoplamento, o tamanho do código, sua complexidade e etc.
%	   \item humanos: estrutura de comunicação, familiaridade com o sistema, níveis de habilidade, liderança e etc. 
%      \end{itemize}
      
      
      %\item Yet most of the research studies have paid little attention to how software engineers understand the system and the information needed to perform a maintenance task. 
      \item Ao analisar a manutenibilidade deve-se observar três dimensões\cite{Hanafi2015}:
      \begin{itemize}
	     \item As \alert{pessoas} que executam manutenção de software.
	     \item Os \alert{objetivos} e as \alert{tarefas} da manutenção.
	     \item \underline{As \alert{propriedades} técnicas do sistema em consideração}
      \end{itemize}
      
    \end{itemize}
\end{frame}

\begin{frame}[t, fragile]{Desafios Específicos}
    \begin{itemize}
      \item A \alert{manutenção} em linhas de produtos é considerada mais \alert{complexa} do que em \alert{sistemas tradicionais} $\rightarrow$ \alert{modificações} em um módulo podem afetar \alert{diversos produtos}.
      \item A \alert{quantidade} de ferramentas e guidelines para auxiliar a manutenção ou aumentar a manutenibilidades da linhas são \alert{limitadas}  \cite {Vale2015}.
      
    \end{itemize}
\end{frame}