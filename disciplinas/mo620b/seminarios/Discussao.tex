\section{Discussão dos Artigos}
\begin{frame}[t, fragile]{Discussão do Artigo 1}
    \begin{itemize}
 
      \item Os guidelines foram \alert{validados} em uma \alert{única} linha de produtos desenvolvida para fins acadêmicos
      
%       (Talvez isso torne mais críticos a contribuição negativa dos critérios apresentados) 
       \item O enunciado do critério C1 \alert{não está claro} no artigo. Ele não contemplaria o C7 ?       
      \item O enunciado dos critérios C2 e C5 não são \alert{objetivamente} verificáveis. 
%       \begin{itemize}
% 	\item o que se entende por "\underline{grande} ou \underline{muita} quantidade de responsabilidades de classes e métodos" ?.
%       \end{itemize}
       
      \item Os \textit{guidelines} de \alert{quatro} critérios \alert{reduziram} o MI da linha de produto estudada e para \alert{um} critério
	foi \alert{inócuo}. 
%       \begin{itemize}
% 	\item sugere-se a não adequação dessas orientações para linhas ou
% 	\item que MI deva ser ajustado para esse domínio.
%       \end{itemize}
      
     
      %Já que estudos apontam que a simples presença de linhas de comentários não contribuiem positivamente para a manutenibilidade.
    \end{itemize}
\end{frame}

\begin{frame}[t, fragile]{Discussão do Artigo 2}
    \begin{itemize}
     \item A utilização de critérios \alert{subjetivos} na coleta de dados é um ponto discutível.
     \item A compreensibilidade \alert{não está} definida na norma ISO/IEC 9126 como \alert{sugere} o artigo.
    \end{itemize}
\end{frame}

\begin{frame}[allowframebreaks, t, fragile]{Discussão do Artigo 3}
    \begin{itemize}
      \item Não apresenta as razões para a escolha das  \alert{duas linhas} (desenvolvidas com técnicas diferentes) para \alert{validação dos resultados}.
      \item O artigo utilizou o termo \alert{"instabilidade"} ao invés do \alert{"estabilidade"} como está na norma \alert{ISO/IEC 9126}.      
    \end{itemize}
\end{frame}