\section{Manutenilidade}
\subsection{Conceitos}
\begin{frame}[allowframebreaks, t, fragile]{O Que é Manutenibilidade ?}
    \begin{itemize}
      \item Em primeiro lugar, não existe um entendimento comum sobre o que é manutenibilidade, como
        ela pode ser atingida, medida e avaliada.
      \begin{itemize}
        \item Diferentemente de outros atributos como desempenho e correção (correctness)
        \item Toda organização de software de tamanho significativo parece ter a sua própria definição
        de tamanho
      \end{itemize}

      \item O glossário de termos da IEEE define \alert{manutenibilidade} assim:
      \begin{itemize}
        \item Facilidade com que o sistema ou componente desse sistema pode ser modificado para corrigir falhas, 
        melhorar o desempenho ou outros atributos, ou adaptar à mudanças no ambiente
      \end{itemize}
       
      \item O Software Engineering Institute define \alert{manutenibilidade} assim:
      \begin{itemize}
        \item O esforço necessário para realizar modificações na implementação de um componente específico.
      \end{itemize}
      
      \item A norma ISO/IEC 9126 define \alert{manuteniblidade} assim:
      \begin{itemize}
        \item Esforço necessário para fazer modificações específicas de software. Tem como subcaracterísticas 
        \begin{itemize}
          \item \textbf{analisabilidade}: medida de esforço necessário para diagnosticar deficiências ou causas de 
            falhas, ou localizar as partes a serem modificadas para corrigir os problemas
          \item \textbf{modificabilidade}: medida de esforço necessário para realizar alterações, remover falhas ou 
            para adequar o produto a eventuais mudanças de ambientes operacionais
          \item \textbf{estabilidade}: medida do risco de efeitos inesperados provenientes de modificações
          \item \textbf{testabilidade}: medida de esforço necessário para testar o software alterado
          \item conformidade
        \end{itemize}
      \end{itemize}
      
    \end{itemize}
\end{frame}

\subsection{Importância}
\begin{frame}[allowframebreaks, t, fragile]{Por que é Importante ?}
   \begin{itemize}
    \item Virtualmente quaisquer organização dependente de software tem vital interesse na redução das suas 
      atividades de manutenção
    
    \item Manutenibilidade é amplamente aceita como um importante atributo de qualidade de sistemas de software
      em razão do seu impacto econômico.
      
    \item Manutenibilidade de software tornou-se uma das mais importantes preocupações da indústria de software
    \item F.Brooks reivindica "O custo total de manter um software é tipicamente 40 porcento ou mais do custo de desenvolvimento"

    \item Parik tem uma visão mais pessimista, reivindicando que de 45 a 60 é gasto com manutenção".
    \item Corbi and Yourdon reivindica que a manutenibilidade de software e um dos maiores desafios para da década de 1990
    
   \end{itemize}
\end{frame}

\subsection{Modelos de Avaliação da Manutenibilidade}
\begin{frame}[allowframebreaks, t, fragile]{Modelos de Avaliação da Manutenibilidade}
   \begin{itemize}
    \item \textbf{Modelo de avaliação multidimensional}:   
    \item \textbf{Modelo de regressão polinominal}: 
    \begin{itemize}
      \item Explora o relacionamento entre a manutenibilidade de software e as métricas de software. 
    \end{itemize}   
    \item \item \textbf{Medida de agregação de complexidade}:
    \item \textbf{Análise de componentes principais}:
    \item \textbf{Análise de fator}
   \end{itemize}
\end{frame}

\subsection{Manutenção versus Manutenibilidade}
\begin{frame}[t, fragile]{Manutenção versus Manutenibilidade}
    blabla
\end{frame}


