\section{Related Work}\label{sec:relatedwork} 

To the extent of our knowledge, only two papers\cite{Cavalcanti:2014, Uddin:2017} have reviewed the literature about bug report severity prediction. Cavalcanti et al.\cite{Cavalcanti:2014} performed a review of 142 papers, published between 2000 to 2012, that investigated challenges and opportunities for software change repositories. Just seven of them are related to change request prioritization, which is defined in Bugzilla by two fields: \textit{priority} and \textit{severity}. Four out of them addressed bug report severity prediction. Only two papers (Lamkanfi et al.\cite{Lamkanfi:2010} and Lamkanfi et al.\cite{Lamkanfi:2011}), however, addressed severity prediction on FLOSS projects. That mapping review shows that all of seven papers used some Information Retrieval Model: one paper used the vector space representation (binary and term count), and all seven used TF-IDF. It also shows that all papers implement learning techniques such as SVM (4 out of 7), Decision tree (2 out of 7), k-NN (2 out of 7), Naïve Bayes (3 out of 7), and Naïve Bayes Multinomial (1 out of 7).

The review presented by Uddin et al. \cite{Uddin:2017}, in turn, surveyed published papers in bug prioritization. The authors reviewed and analyzed in depth 32 distinct papers published between 2003 and 2015. The aim of that analysis was ``to summarize the existing work on bug prioritization and some problems in working with bug prioritization". That work categorizes research initiatives according to ML algorithms, evaluation measures, data sets, researchers, and publication venue. It can be worth to note that the authors investigated only eight papers about predicting of severity level. In contrast, the current mapping review investigated in depth 27 papers about bug report severity prediction published from 2010 to 2017.

Although these reviews present relevant results for research in this area, they have one difference to our work. There is no explicit focus on bug report severity prediction. In fact, those papers perform a brief review of this topic by addressing mainly bug report priority prediction. The current review has a broader goal of mapping studies addressing many research questions and concepts related to bug report severity prediction. Furthermore, this review considers relevant aspects (e.g., sampling techniques and statistical tests), not addressed before, in the characterization of papers.